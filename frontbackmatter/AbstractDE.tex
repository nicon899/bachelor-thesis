%*******************************************************
% Abstract in German
%*******************************************************
\begin{otherlanguage}{ngerman}
	\pdfbookmark[0]{Zusammenfassung}{Zusammenfassung}
	\chapter*{Zusammenfassung}
Die vorliegende Arbeit beschäftigt sich mit der Frage, wie ein on-premise BI-System, bestehend aus Anwendungen für Datenintegration, Speicherung und Reporting, in die Azure Cloud migriert werden kann. Zu diesem Zweck wird eine Cloud BI-Architektur entworfen, die alle benötigen Funktionalitäten abgedeckt und dabei möglichst kostengünstig ist. Dazu werden in Bezug auf eine Anforderungsspezifikation verschiedene Azure Dienste evaluiert und basierend auf den Ergebnissen ausgewählt. Damit die Anforderungen an Sicherheit und Datenschutz erfüllt werden, wird die Architektur um dafür geeignete Dienste ergänzt.
Um zu zeigen, dass das Cloud BI-Konzept in der Praxis funktioniert, wird die neue BI-Architektur als Prototyp umgesetzt. Bei der Umsetzung wird darauf eingegangen, wie eine Migration mit möglichst geringem Aufwand durchgeführt werden kann. Eine abschließende Auswertung der Testfälle zeigt, dass der Prototyp mehr Anforderungen als das Bestandssystem erfüllt. Auf dieser Basis wird diskutiert, wann sich der Umstieg von on-premise zu Cloud BI lohnt und was dabei zu berücksichtigen ist.
\end{otherlanguage}