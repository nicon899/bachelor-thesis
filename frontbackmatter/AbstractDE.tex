%*******************************************************
% Abstract in German
%*******************************************************
\begin{otherlanguage}{ngerman}
	\pdfbookmark[0]{Zusammenfassung}{Zusammenfassung}
	\chapter*{Zusammenfassung}
Business Intelligence (BI) wird in Unternehmen zum Analysieren und Visualisieren von Daten genutzt. Das hier betrachtete BI-System besteht aus einem relationalen Data Warehouse, OLAP Cubes, Data Marts und Reports auf Basis eines Webframeworks. Dieses System soll durch eine moderne BI-Lösung in der Azure Cloud ersetzt werden. Hierbei gibt es eine Vielzahl an Architekturvarianten mit unterschiedlichen Vor- und Nachteilen. Daraus resultiert die Frage, wie ein Cloud BI-System aufgebaut sein sollte, damit es möglichst kostengünstig die gewünschte Funktionalität abdeckt und gleichzeitig Anforderungen an Sicherheit und Datenschutz genügt. Zum Beantworten dieser Frage werden verschiedene Komponenten der Azure Cloud evaluiert. Basierend auf den Evaluationsergebnissen wird ein vollständiges Konzept für eine neue BI-Architektur entworfen. Das neue Konzept wird als Prototyp umgesetzt, an dem verschiedene Anwendungsfälle getestet werden können. So wird sichergestellt, dass die Anforderungen in der Praxis erfüllt und mögliche Probleme frühzeitig identifiziert werden. Zum Schluss wird diskutiert, ob sich der Umstieg von on-premise zu Cloud BI lohnt und was dabei berücksichtigt werden sollte.
\end{otherlanguage}