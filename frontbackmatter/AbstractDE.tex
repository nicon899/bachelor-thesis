%*******************************************************
% Abstract in German
%*******************************************************
\begin{otherlanguage}{ngerman}
	\pdfbookmark[0]{Zusammenfassung}{Zusammenfassung}
	\chapter*{Zusammenfassung}
Business Intelligence (BI) wird in Unternehmen zum Analysieren und Visualisieren von Daten genutzt. Dafür sind Anwendungen für Datenintegration, Speicherung und Reporting notwendig. In einem aktuellen BI-System werden diese Anwendungen auf eigener Hardware betrieben und sollen zukünftig durch eine moderne BI-Lösung in der Azure Cloud ersetzt werden. Daraus resultiert die Frage, wie eine Architektur aussehen sollte, die möglichst kostengünstig alle benötigen Funktionalitäten abgedeckt und gleichzeitig Anforderungen an Sicherheit und Datenschutz gerecht wird. Zum Beantworten dieser Frage werden verschiedene Komponenten der Azure Cloud in Bezug auf eine Anforderungsspezifikation evaluiert. Die Ergebnisse werden verwendet, um ein vollständiges Konzept für eine neue BI-Architektur zu entwerfen. Zusätzlich werden beim Entwurf weitere Dienste zum Optimieren der Sicherheit berücksichtigt. Die neue BI-Architektur wird als Prototyp umgesetzt, der zeigt, dass das Cloud-BI Konzept in der Praxis funktioniert. Bei der Umsetzung wird darauf eingegangen, wie eine mögliche Migration mit möglichst geringen Aufwand durchgeführt werden könnte. Eine abschließende Auswertung der Testfälle zeigt, dass der Prototyp mehr Anforderungen als das Bestandssystem erfüllt. Auf dieser Basis wird diskutiert, wann sich der Umstieg von on-premise zu Cloud BI lohnt und was dabei berücksichtigt werden sollte.
\end{otherlanguage}