\section{Evaluation von Azure Diensten} \label{sec:evaluation}
Anhand der im vorherigen Abschnitt beschriebenen Anforderungen sollen im Folgenden die Azure Dienste evaluiert werden. Die Evaluation stellt die Grundlage für den Entwurf der Cloud BI-Architektur dar. Die untersuchten Dienste werden jeweils kurz vorgestellt und anschließend wird beschrieben, welche Anforderungen diese erfüllen können. 

\subsection{Azure Logic Apps} \label{sec:grundlagen:azure_dienste:logicApps}
Mit diesem Dienst können automatisierte Workflows zur Datenintegration erstellt werden. Die Workflows sind sehr anpassungsfähig und es steht eine Vielzahl an vorgefertigten Konnektoren zur Verfügung \cite{kumar_serverless_2019}.

\subsection{Azure Data Factory} \label{sec:grundlagen:azure_dienste:dataFactory}
Data Factory ist ein Datenintegrationsdienst, der für on-premise und Cloud Systeme genutzt werden kann. Der Dienst wurde speziell für den gemeinsamen Einsatz mit anderen Diensten entwickelt und übernimmt dabei die Bewegung, Transformation und Verarbeitung der Daten, zwischen den unterschiedlichen Systemen \cite{klein_iot_2017}.

\subsection{Azure SQL} \label{sec:grundlagen:azure_dienste:sql}
Azure SQL ist eine relationale Datenbank, die vergleichbar mit dem on-premise Microsoft SQL Server, der im aktuellen BI-System genutzt wird, ist. Bei einer Migration hat dies den Vorteil, dass die meisten SQL-Queries zum Laden und Transformieren von Daten, ohne Anpassung verwendet werden können. Der Einsatz von Azure SQL empfiehlt sich besonders, wenn hochgradig relationale Daten vorliegen \cite{reagan_azure_2018}. 

Die alternativen relationalen Datenbanken in Azure, wie MySQL und PostgresSQL, werden in dieser Arbeit nicht betrachtet.

\subsection{Table Storage} \label{sec:grundlagen:azure_dienste:tableStorage}
Der NoSQL-Dienst Table Storage gehört zu den günstigsten Speichermöglichkeiten in Azure. Die Daten werden als Schlüssel/Wert-Paare in Tabellen gespeichert, diese sind jedoch nicht relational und können nicht miteinander gejoint werden. Eine Tabelle besteht aus einer oder mehreren Partitionen und eine Partition besteht aus ein oder mehreren Zeilen. Auf jeden Tabelleneintrag kann über einen eindeutigen Schlüssel zugegriffen werden. Dieser ist eine Kombination aus Partitions- und Zeilen-Schlüssel. Die Partitionierung über mehrere Server ermöglicht es Datenmengen im dreistelligen Terabyte Bereich zu speichern \cite{reagan_azure_2018}. 

\subsection{Azure Cosmos DB} \label{sec:grundlagen:azure_dienste:cosmosDB}
Bei diesem Dienst handelt es sich um eine Datenbank, die mehrere Datenmodelle, wie Schlüssel-Wert, Dokument oder Graph, unterstützt. Eines der Features ist die automatisierbare Replikation über verschiedene Regionen. Dadurch können weltweit geringe Latenzzeiten gewährleistet werden. Die \acp{sla} sichern außerdem eine Verfügbarkeit von 99,99\% zu \cite{guay_paz_introduction_2018}. 

\subsection{Azure Data Lake Gen 2} \label{sec:grundlagen:azure_dienste:dataLake}
\acp{blob} sind unstrukturierte Dateien, die sich nicht zum Speichern in einer Datenbank eignen. Eine kostengünstige Möglichkeit, diese in der Cloud abzulegen, ist der Blob Storage. Auf diesen Dienst baut auch der Data Lake Gen 2 auf, welcher für die Analyse der gespeicherten Daten spezialisiert wurde \cite{soh_azure_2020}.

\subsection{Azure Data Lake Analytics} \label{sec:grundlagen:azure_dienste:dataLakeAnalytics}
Dieser Dienst ist für eine möglichst einfache und kosteneffektive Big Data Anaylse ausgelegt. Für die Analysen wird die Abfragesprache U-SQL (Unified SQL) verwendet, welche die Stärken von SQL und C\# miteinander vereint \cite{klein_iot_2017}. 

\subsection{Azure HDInsight} \label{sec:grundlagen:azure_dienste:hdInsight}
Bei HDInsight handelt es sich um Apache Hadoop, als vollständig verwalteten Cloud Dienst. Hadoop ist eine Sammlung von Open~=Source Komponenten, die für die verteilte Verarbeitung und Analyse großer Datensätze konzipiert wurden. Dieser Dienst ist vergleichsweise komplex, da individuelle Konfigurationen notwendig sind \cite{klein_iot_2017}.

\subsection{Azure Databricks} \label{sec:grundlagen:azure_dienste:databricks}
Databricks setzt Apache Spark als Cloud-Dienst um. Letzteres ist ein weiteres Open-Source Framework für die Big Data Analyse. Im Gegensatz zu Hadoop werden die Daten zur schnelleren Verarbeitung im Arbeitsspeicher vorgehalten. Ein weiteres Feature ist die mögliche Echtzeitverarbeitung von Streaming-Daten \cite{soh_data_2020}.

\subsection{Azure Machine Learning} \label{sec:grundlagen:azure_dienste:machineLearning}
Azure Machine Learning ist eine Sammlung von Diensten und Tools für maschinelles Lernen in der Cloud \cite{soh_data_2020}.

\subsection{Azure Synapse Analytics} \label{sec:grundlagen:azure_dienste:synapseAnalytics}
Synapse Analytics vereint die Datenintegration, Data Warehousing und Big Data Analysen zu einem Dienst. Die Datenintegration verwendet die Data Factory, wird hier aber als \textit{Synapse Pipeline} bezeichnet. Die integrierten Daten können unabhängig von ihrer Struktur gespeichert werden. Durch die Unterstützung von Apache Spark wird die Analyse von großen Datenmengen ermöglicht \cite{shiyal_beginning_2021}.

\subsection{Azure Analysis Services} \label{sec:grundlagen:azure_dienste:analysisServices}
Dieser Dienst kann als Semantik-Schicht zwischen \ac{dwh} und Endnutzer agieren. In dieser Funktion können beispielsweise Spalten umbenannt oder irrelevante Werte ausgeblendet werden, damit übersichtliche Reports entstehen. Auch eine granulare Implementierung von \ac{rbac} wird so vereinfacht. Der Dienst kann jedoch nur mit tabellarischen Datenmodellen umgehen, nicht mit multidimensionalen \cite{how_beyond_2020}.

\subsection{Power BI} \label{sec:grundlagen:azure_dienste:powerBI}
Power BI ist Microsofts Dienst zum Visualisieren von Daten. Es werden aber auch \ac{etl} Prozesse und Datenanalyse auf Basis von Analysis Services bereitgestellt. Mit Power BI können vorgefertigte und Self-Service Reports erstellt werden \cite{how_beyond_2020}.

\subsection{Azure Purview} \label{sec:grundlagen:azure_dienste:purview}
Purview dient der zentralen Verwaltung der Datengovernance, sowohl in Cloud- als auch on-premise Umgebungen. Dies ermöglichst beispielsweise, mit wenig Aufwand Datenquellen aufzufinden \cite{lesteve_purview_2021}.