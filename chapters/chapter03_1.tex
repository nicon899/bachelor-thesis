\section{Anforderungen}
\label{sec:anforderungen}
Im Folgenden soll ein kurzer Überblick über die wichtigsten Anforderungen gegeben werden. Eine vollständige Anforderungsspezifikation kann in Appendix~\ref{ch:anforderungsspezifikation} gefunden werden. 

\subsection{Funktionale Anforderungen}
Es sollen Daten aus verschiedenen Quellsystemen integriert werden können, unabhängig davon, ob diese on-premise oder in der Cloud betrieben werden. Die extrahierten Daten müssen langfristig gespeichert werden. Auf den gespeicherten Daten sollen automatisierte Analysen und Auswertungen durchgeführt werden können. Auch die Anwendung von Data-Science mit den Programmiersprache R und Python soll unterstützt werden. Die Ergebnisse der Auswertungen sollen visualisiert in Reports angezeigt werden. Neben den fest definierten Auswertungen sollen auch Self-Service-Reports bereitgestellt werden. Aus Datenschutzgründen sollen alle Übertragungen und Verarbeitungen der Daten automatisch dokumentiert werden. Für die Zugriffskontrolle muss es außerdem möglich sein, alle Daten zu klassifizieren.

\subsection{Nicht-funktionale Anforderungen}
Zu den wichtigsten Anforderungen an Cloud-Systeme gehören die Sicherheit und der Datenschutz \cite{gurjar_cloud_2013}. Daher soll das neue System bestmöglich vor Angriffen geschützt sein. Daneben ist eine zuverlässige Authentifizierung und Autorisierung der Nutzer unumgänglich. Für alle Daten und Informationen muss sichergestellt werden, dass nur berechtigte Personen diese sehen können. Worauf ein Mitarbeiter Zugriff haben darf, ist abhängig von seiner Rolle im Unternehmen und seinem regionalen Standort. Ein anderes Kriterium ist die Verwendung von möglichst wenig neuen oder unterschiedlichen Technologien. Dadurch soll der Wartungsaufwand und die Anzahl der hierfür notwendigen Personen verringert werden.