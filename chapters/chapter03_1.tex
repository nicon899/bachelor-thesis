\section{Anforderungen}
\label{sec:anforderungen}
Die Anforderungen an das Cloud-BI-System werden aus der Funktionalität des bestehenden Systems abgeleitet, aber auch um weitere ergänzt.

\subsection{Funktionale Anforderungen}
\label{subsec:funktionaleAnforderungen}
Es sollen Daten aus verschiedenen Quellsystemen integriert werden können, unabhängig davon, ob diese on-premise oder in der Cloud betrieben werden. Die extrahierten Daten müssen langfristig gespeichert werden. Auf den gespeicherten Daten sollen automatisierte Analysen und Auswertungen durchgeführt werden können. Auch komplexere Analysen mit der Programmiersprache R sollen unterstützt werden. Die Ergebnisse der Auswertungen sollen visualisiert in Reports angezeigt werden. Neben den fest definierten Auswertungen soll auch Ad-hoc-Reporting unterstützt werden.

\subsection{Nicht-Funktionale Anforderungen}
\label{subsec:NichtfunktionaleAnforderungen}
Zu den wichtigsten Anforderungen an Cloud-Systeme gehören die Sicherheit und der Datenschutz \cite{gurjar_cloud_2013}. Daher soll das neue System bestmöglich vor Angriffen geschützt sein. Daneben ist eine zuverlässige Authentifizierung und Autorisierung der Nutzer unumgänglich. Für alle Daten und Informationen muss sichergestellt werden, dass nur berechtigte Personen diese sehen können. Worauf ein Mitarbeiter Zugriff haben darf, ist abhängig von seiner Rolle im Unternehmen und seinem regionalen Standort.