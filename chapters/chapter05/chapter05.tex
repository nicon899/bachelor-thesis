\chapter{Diskussion} \label{ch:diskussion}
Bevor eine tatsächliche Migration des BI-Systems in die Cloud durchgeführt wird, sollten einige Aspekte genauer betrachtet werden. Es können viele Vor- und Nachteile von Cloud gegenüber on-premise BI gefunden werden. Zu den Vorteilen zählt die kürzere Entwicklungszeit, die nahezu unbeschränkt hochskalierbare Rechenleistung und Speicherkapazität und die potenziellen Kostenersparnisse \cite{ouf_cloud_2011}. Zu den Nachteilen gehört die Abhängigkeit von dem Cloud-Provider und das notwendige Vertrauen in diesen. Auf der anderen Seite sind diese Bedenken oft unbegründet und die Umgebung des Cloud-Providers ist besser gesichert, als die eigene \cite{menon_business_2012}. Allerdings wurde im Rahmen dieser Arbeit festgestellt, dass besonders die Umsetzung einer sicheren Netzwerkinfrastruktur, zu einem komplexeren Gesamtsystem führt. Beispielsweise folgt aus dem Schutz des \ac{dwh} durch eine Firewall, dass \acp{vm} im gleichen virtuellen Netzwerk notwendig sind, damit Power BI und Purview trotzdem darauf zugreifen können. Auf dem \acp{vm} müssen Anwendungen installiert sein, was bedeutet, dass zusätzlicher Aufwand für regelmäßige Aktualisierungen anfällt. Insgesamt ist der Wartungsaufwand für Cloud BI also deutlich größer als ursprünglich erwartet.

Ein anderer Kostenfaktor sind die Preise der Azure Dienste, die in dieser Arbeit nicht im Detail betrachtet wurden. Beim Bereitstellen der Infrastruktur (siehe Abschnitt~\ref{sec:infra}) wurde lediglich darauf geachtet, dass Ressourcenkonfigurationen so gewählt werden, wie es laut Literatur, für das hier betrachtete Szenario am günstigsten ist. Besonders die vielen verschiedenen Preismodelle für die verwendeten Dienste erschweren es, die zu erwartenden Kosten zu bestimmen. Azure bietet zu diesem Zweck einen Preisrechner an. Dort können für die verschiedenen Dienst, die geplanten Leistungs- und Nutzungsdaten angegeben werden, die für das jeweilige Preismodell relevant sind. Die Genauigkeit der Kostenkalkulation ist maßgeblich von der Korrektheit dieser Angaben abhängig \cite{modi_azure_2020}. Aus diesem Grund sollte vor der Migration eine genaue Analyse der benötigten Leistung für das Cloud BI-System durchgeführt werden.

Wenn die Kostenfrage beantwortet ist, bietet es sich an, das zu erwartende \ac{roi} zu ermitteln. Das \ac{roi} beschreibt die Vorteile, die mit dem eingesetzten Geld erzielt werden. Das können beispielsweise die Kostenersparnisse sein, dadurch dass keine eigene Hardware mehr benötigt wird. Wird das \ac{roi} sowohl für den Weiterbetrieb von on-premise BI als auch der Migration zu Cloud BI ermittelt, ist dieses eine hilfreiche Metrik, um zu entscheiden, ob sich der Umstieg rentieren würde. Dazu können verschiedene \ac{roi} Metriken kalkuliert und für beide Szenarien gegenüber gestellt werden \cite{menon_business_2012}. Für den Fall, dass das Ergebnis aus finanzieller Sicht gegen die Migration zu Cloud BI spricht, sollte noch die Tabelle~\ref{table:auswertung} berücksichtigt werden. Dann kann basierend auf der Anforderungsspezifikation entschieden werden, ob die zusätzlichen Kosten gegebenenfalls gerechtfertigt wären. Sind sie das nicht, sollte keine Migration zu dem vorgestellten BI-System durchgeführt werden. Stattdessen sollte entweder die Architektur überarbeitet werden, zum Beispiel, in dem auf einzelne Anforderungen und Dienste verzichtet wird, oder es sollten alternative Cloud Provider in Erwägung gezogen werden. Wenn vorerst keine weitere Untersuchungszeit in dieses Thema investiert werden soll, wäre dies spätestens vor der nächsten größeren Anschaffung für das on-premise BI-System zu empfehlen. Denn der Vergleich des \ac{roi} könnte zu einem anderen Ergebnis kommen, wenn für das Bestandssystem neue leistungsfähigere Hardware angeschafft werden muss.

Insgesamt wird jedoch vermutet, dass sich schon vorher für den Umstieg zu Cloud BI entschieden wird. Dann sollte ein weiteres Risiko, nämlich das mögliche Scheitern, betrachtet werden. Dank der Ergebnisse dieser Arbeit wird davon ausgegangen, dass das Risiko für eine fehlschlagende Migration gering ist. Zum einen wurde beim Entwurf der Architektur auf die Verfügbarkeit der ausgewählten Dienste geachtet. Das gilt sowohl für die Ausfallzeit, als auch die voraussichtliche Lebensdauer. Zum Anderen wurde die Migration bereits mit einem Prototyp getestet und ein mögliches Vorgehen wurde in Kapitel~\ref{ch:praktischeUmsetzung} vorgestellt. Dabei konnten keine technischen Aspekte festgestellt werden, die gegen einen erfolgreichen Umstieg sprechen würden. Stattdessen konnte zusätzlich gezeigt werden, wie der Migrationsaufwand möglichst gering gehalten werden kann. Für zusätzliche Sicherheit könnte auch darüber nachgedacht werden, beide Systeme für einen Übergangszeitraum parallel zu betreiben. 
 
 Zusammenfassend ist es erforderlich, die konkreten Kosten des Cloud BI-Systems, unter Berücksichtigung des Wartungsaufwands, und damit das zu Erwartende \ac{roi} zu ermitteln, damit entschieden werden kann, ob sich eine Migration lohnen würde. Wird allerdings nur die Anforderungsspezifikation betrachtet, ist es auf jeden Fall zu empfehlen, zu den vorgestellten Cloud BI-System umzusteigen.