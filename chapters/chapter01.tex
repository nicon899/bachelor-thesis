\chapter{Einleitung}
\label{ch:intro}
Für Unternehmen hat es einen hohen Stellenwert, Entscheidungen auf Basis von aktuellen, korrekten und qualitativ hochwertigen Informationen treffen zu können. Denn so kann das Risiko für Fehlentscheidungen minimiert werden \cite{grunwald_business_2009}. Mit einem BI-System können diese wertvollen Informationen aus operativen Daten gewonnen werden \cite{muller_business_2013}.

\section{Motivation}
\label{sec:intro:motivation}
Das hier betrachtete BI-System wird aktuell on-premise betrieben. Im Rahmen einer Migration zu einem neuen Kundensupporttool wurde es jedoch um Ressourcen aus der Azure Cloud erweitert. Dazu zählen eine SQL-Datenbank, ein Storage Account und das Analysetool PowerBI. Bei dieser Erweiterung lag der Fokus darauf, möglichst schnell alle Reporting Funktionalitäten, die für das alte Kundensupporttool vorhanden waren, auch für das neue bereitzustellen. Bei der daraus resultierten Architektur besteht daher Optimierungspotential. Besonders die Themen Sicherheit, Datenschutz, Kosten und Wartungsaufwand sollten genauer betrachtet werden. Dieser bereits existierende Teil des BI-Systems in der Cloud wird im Folgenden nicht weiter betrachtet. Erwähnenswert ist dieses Projekt, weil dadurch ein erster Eindruck gewonnen werden konnte, welche Möglichkeiten BI in der Cloud bietet. Daraus folgt die Frage, wie das gesamte on-premise BI-System durch eine moderne Alternative in der Azure Cloud ersetzt werden könnte. Zur Beantwortung dieser Frage ist besonders der Entwurf einer geeigneten Architektur von Interesse. Diese soll nicht nur die Funktionalität des aktuellen BI-Systems abdecken, sondern es sollen auch fortgeschrittene Analysetechniken aus dem Bereich Data Science angewendet werden können. Hierfür liegen bereits fertige Skripte in der Programmiersprache R vor, die jedoch nie in das on-premise BI-System integriert wurden.

Gründe für den Umstieg auf eine Cloud BI-Lösung gibt es viele. Zum Beispiel ist weder eigene Hardware, noch das Installieren und Aktualisieren von Software notwendig. Stattdessen kann auf alle BI-Komponenten direkt über den Webbrowser zugegriffen werden. Ein weiterer Vorteil ist die Skalierbarkeit. In einem klassischen BI-System werden üblicherweise strukturierte Daten in einem DWH gespeichert, bevor diese analysiert werden können. Heutzutage ist ein Großteil der Daten, wie Bilder, Logfiles oder Sensordaten, jedoch unstrukturiert. Durch die unbegrenzt hochskalierbare Rechenleistung in der Cloud wird es Unternehmen ermöglicht, diese unstrukturierten Daten effizient zu analysieren und in Reports zu verwenden, ohne dass dafür unverhältnismäßige Kosten entstehen.\cite{gurjar_cloud_2013}. Denn bei hohem Rechenbedarf können zusätzliche Ressourcen automatisch hinzugeschaltet werden. Werden diese nicht mehr benötigt, können sie abbestellt werden und müssen nicht mehr bezahlt werden. Im Vergleich zu einem System, das immer auf die maximale Auslastung vorbereitet ist, können so hohe Kosten gespart werden \cite{ouf_cloud_2011}.
% \section{Ziel der Arbeit}
% \label{sec:intro:goal}


\section{Gliederung}
\label{sec:intro:structure}
Zu Beginn werden anhand des aktuellen BI-Systems die wichtigsten Grundlagen erläutert und es werden die neuen Möglichkeiten in der Azure Cloud vorgestellt. Anschließend wird beschrieben, wie die neue Cloud BI-Architektur entworfen werden soll und wie eine mögliche Migration ablaufen könnte. Der folgende Abschnitt beschäftigt sich mit der Entwicklung eines Prototyps, mit dem die wichtigsten Anwendungsfälle getestet werden können. Danach wird auf Basis der neu gewonnenen Erkenntnisse und bestehender Literatur diskutiert, wie sinnvoll der tatsächliche Umstieg von on-premise zu Cloud-BI wäre. Abschließend werden die Ergebnisse zusammengefasst und ein Ausblick gegeben, welche Bedeutung diese zukünftig haben könnten.