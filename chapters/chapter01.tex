\chapter{Einleitung} \label{ch:intro}
Die Ausgangslage für diese Arbeit ist ein on-premise betriebenes BI-System der Software AG. Im Rahmen einer Migration zu einem neuen Kundensupporttool wurde dieses jedoch um Ressourcen aus der Azure Cloud erweitert. Dazu zählen eine SQL~=Datenbank, ein Blob~=Storage und das Analysetool PowerBI. Bei dieser Erweiterung lag der Fokus darauf, möglichst schnell alle Reporting Funktionalitäten, die für das alte Kundensupporttool vorhanden waren, auch für das neue bereitzustellen. Bei der daraus resultierten Architektur besteht daher Optimierungspotential. Besonders die Themen Sicherheit, Datenschutz, Kosten und Wartungsaufwand sollten genauer betrachtet werden. Dieser bereits existierende Teil des BI-Systems in der Cloud wird im Folgenden nicht weiter betrachtet. Erwähnenswert ist dieses Projekt, weil dadurch ein erster Eindruck gewonnen werden konnte, welche Möglichkeiten BI in der Cloud bietet.

\section{Motivation} \label{sec:intro:motivation}
Für Unternehmen hat es einen hohen Stellenwert, Entscheidungen auf Basis von aktuellen, korrekten und qualitativ hochwertigen Informationen treffen zu können. Denn so kann das Risiko für Fehlentscheidungen minimiert werden \cite{grunwald_business_2009}. Mit einem BI-System können diese wertvollen Informationen aus operativen Daten gewonnen werden \cite{muller_business_2013}.

Gründe für den Umstieg auf eine Cloud BI-Lösung gibt es viele. Zum Beispiel ist weder eigene Hardware, noch das Installieren und Aktualisieren von Software notwendig. Stattdessen kann auf alle BI-Komponenten direkt über den Webbrowser zugegriffen werden. Ein weiterer Vorteil ist die Skalierbarkeit. In einem klassischen BI-System werden üblicherweise strukturierte Daten in einem \ac{dwh} gespeichert, bevor diese analysiert werden können. Heutzutage ist ein Großteil der Daten, wie Bilder, Logfiles oder Sensordaten jedoch unstrukturiert. Durch die unbegrenzt hochskalierbare Rechenleistung in der Cloud wird es Unternehmen ermöglicht, diese unstrukturierten Daten effizient zu analysieren und in Reports zu verwenden, ohne dass dafür unverhältnismäßige Kosten entstehen \cite{gurjar_cloud_2013}. Denn bei hohem Rechenbedarf können zusätzliche Ressourcen automatisch hinzugeschaltet werden. Werden diese nicht mehr benötigt, können sie abbestellt werden und müssen nicht mehr bezahlt werden. Im Vergleich zu einem System, das immer auf die maximale Auslastung vorbereitet ist, können so hohe Kosten gespart werden \cite{ouf_cloud_2011}.

\section{Ziel der Arbeit} \label{sec:intro:ziele}
Ziel dieser Arbeit ist der Entwurf einer modernen Cloud BI-Architektur, die als vollwertiger Ersatz für das vorhandene BI-System dienen kann. Die hier entworfene Architektur wird als Prototyp umgesetzt. Durch das Testen des Prototyps soll sichergestellt werden, dass die Architektur den Anforderungen in der Praxis genügt. Des Weiteren soll ein möglicher Migrationsprozess erarbeitet werden.

Die neue BI-Lösung soll nicht nur die benötigte Funktionalität des aktuellen BI-Systems abdecken, sondern auch um neue ergänzt werden. Bei letzteren steht insbesondere die Bereitstellung einer geeigneten Infrastruktur für Data Scientists im Fokus. Diesen soll es mit minimalem Aufwand ermöglicht werden, fortgeschrittene Datenanalysen mit den Programmiersprache R und Python durchzuführen.

\section{Gliederung} \label{sec:intro:struktur}
Zu Beginn werden verwandte Arbeiten und die wichtigsten Grundlagen, wie der Aufbau des on-premise BI-Systems und zu der Azure Cloud, erläutert. Das folgende Kapitel beschäftigt sich mit dem Entwurf einer neuen Cloud BI-Architektur. Dabei werden zunächst ausgewählte Azure Dienste vorgestellt und evaluiert. Darauf basierend wird ein vollständiges Konzept für eine Cloud BI-Lösung entworfen. Als Nächstes wird die Entwicklung eines Prototyps beschrieben, an dem die wichtigsten Anwendungsfälle getestet werden. In diesem Zusammenhang wird außerdem auf ein mögliches Vorgehen bei der Migration eingegangen. Danach wird anhand bestehender Literatur und der neu gewonnenen Erkenntnisse diskutiert, wie sinnvoll der tatsächliche Umstieg von on-premise zu Cloud-BI wäre und was dabei zu beachten ist. Abschließend werden die Ergebnisse zusammengefasst und es wird ein Ausblick gegeben, welche Bedeutung diese zukünftig haben könnten.