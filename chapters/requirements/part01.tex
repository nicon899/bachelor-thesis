\section{Rahmenbedingungen} 

\subsection{Beschränkung auf die Microsoft Azure Cloud} \label{sec:anforderungsspezifikation:azureCloud}
\textbf{Beschreibung}: Abgesehen von den Quellsystemen sollen ausschließlich Dienste der Azure Cloud verwendet werden.
\newline \textbf{Begründung}: Microsoft Azure ist der einzige Cloud-Provider, der aktuell von der Software AG zugelassen ist. Denn nur für Azure liegen die notwendigen Vereinbarungen und Verträge, für die Speicherung und Verarbeitung der Unternehmensdaten vor.

\subsection{Projektdauer von 3 Monaten} \label{sec:anforderungsspezifikation:projektDauer}
\textbf{Beschreibung}: Die Projektdauer ist auf einen Zeitraum von drei Monaten(03.12.2021 bis 03.03.2022) begrenzt.
\newline \textbf{Begründung}: Da dieses Projekt im Rahmen einer Bachelorarbeit durchgeführt wird, ist die Bearbeitungszeit von der Hochschule vorgegeben.

\section{Funktionale Anforderungen} 

\subsection{Datenintegration von on-premise Quellsystemen über Datenbankabfragen} \label{sec:anforderungsspezifikation:datenintegrationOnPremDB}
\textbf{Beschreibung}: In das BI-System sollen Daten aus on-premise Datenbanken integriert werden können.
\newline \textbf{Begründung}: Einige on-premise Quellsysteme speichern ihre Daten in einer eigenen Datenbank. Die dort gespeicherten Daten sollen im BI-System verarbeitet werden können.
\newline \textbf{Testkriterium}: Es ist möglich eine Verbindung zu einer on-premise Datenbank herzustellen, Daten abzufragen und zu speichern.
\newline \textbf{Priorität}: 1

\subsection{Datenintegration von Cloud Quellsystemen über Datenbankabfragen} \label{sec:anforderungsspezifikation:datenintegrationCloudDB}
\textbf{Beschreibung}: In das BI-System sollen Daten aus Cloud Datenbanken integriert werden können.  
\newline \textbf{Begründung}: Einige Cloud Anwendungen speichern ihre Daten in einer eigenen Datenbank. Die dort gespeicherten Daten sollen im BI-System verarbeitet werden können.
\newline \textbf{Testkriterium}: Es ist möglich eine Verbindung zu einem Quellsystem herzustellen, Daten abzufragen und zu speichern.
\newline \textbf{Priorität}: 1

\subsection{Datenintegration von Quellsystemen über REST API} \label{sec:anforderungsspezifikation:datenintegrationREST}
\textbf{Beschreibung}: Es müssen Daten über REST API integriert werden können. 
\newline \textbf{Begründung}: Es gibt Anwendungen, deren Daten nur über REST API zugänglich sind. Auch diese sollen im BI-System verarbeitet werden können.
\newline \textbf{Testkriterium}: REST APIs können aufgerufen und die zurückgelieferten Daten gespeichert werden.
\newline \textbf{Priorität}: 1

\subsection{Dauerhaftes Speichern von Daten (Erhaltung von historischen Daten)} \label{sec:anforderungsspezifikation:dauerhaftesSpeichern}
\textbf{Beschreibung}: Die integrierten Daten sollen dauerhaft im BI-System gespeichert werden.
\newline \textbf{Begründung}: Auswertungen und Analysen sollen historische Daten berücksichtigen können.
\newline \textbf{Testkriterium}: Die historischen Daten sind zu jedem zukünftigen Zeitpunkt abrufbar.
\newline \textbf{Priorität}: 1

\subsection{Datentransformation} \label{sec:anforderungsspezifikation:datentransformation}
\textbf{Beschreibung}: Durch die Transformationen sollen Daten in die benötigte Form umgewandelt werden. Dazu gehört zum Beispiel die Berechnung von Aggregationen, oder das Runden einer Dezimalzahl auf 2 Nachkommastellen.
\newline \textbf{Begründung}: Die Datentransformation ist für einige Auswertungen notwendig.
\newline \textbf{Testkriterium}: Die Daten werden korrekt transformiert.
\newline \textbf{Priorität}: 1

\subsection{Auswertung der gespeicherten Daten} \label{sec:anforderungsspezifikation:datenAuswertung}
\textbf{Beschreibung}: Die gespeicherten Daten sollen ausgewertet werden können. Dazu gehört die Berechnung von Summen oder Durchschnittswerten.
\newline \textbf{Begründung}: Durch die Auswertung sollen Erkenntnisse aus den Daten gewonnen werden, die zum Beispiel als Entscheidungshilfe dienen können.
\newline \textbf{Testkriterium}: Es können Durchschnittswerte und Summen über die Daten berechnet werden.
\newline \textbf{Priorität}: 1

\subsection{Datenanalyse mit Python/R} \label{sec:anforderungsspezifikation:datenanalysePythonUndR}
\textbf{Beschreibung}: Es sollen fortgeschritten Datenanalysen mit den Programmiersprachen Python und R ermöglicht werden. 
\newline \textbf{Begründung}: Durch Data Science können Erkenntnisse gewonnen werden, die mit normalen Analysen nicht möglich sind. Da intern bereits Kenntnisse und Skripte in den Programmiersprachen Python und R existieren, sollen diese eingesetzt werden können.
\newline \textbf{Testkriterium}: Das Durchführen der Analysen führt zu plausiblen Ergebnissen.
\newline \textbf{Priorität}: 2

\subsection{Erstellen und Bereitstellen von Reports} \label{sec:anforderungsspezifikation:reports}
\textbf{Beschreibung}: Die Daten und Analyseergebnisse sollen in Reports visualisiert dargestellt werden.
\newline \textbf{Begründung}: Die Reports sind die Schnittstelle vom Endnutzer zu dem BI-System. Der Endnutzer soll aus den Reports Erkenntnisse gewinnen können, ohne über technische Fähigkeiten zu verfügen.
\newline \textbf{Testkriterium}: Das Erstellen und Betrachten von Reports ist möglich.
\newline \textbf{Priorität}: 1

\subsection{Erstellen und Bereitstellen von Self-Service Reports} \label{sec:anforderungsspezifikation:selfServiceReports}
\textbf{Beschreibung}: Die Daten und Analyseergebnisse sollen in Self-Service Reports visualisiert dargestellt werden.
\newline \textbf{Begründung}: Durch Self-Service Reports können detaillierte bzw. spezialisierte Daten und Analyseergebnisse betrachtet werden, ohne dass für jede Ausprägung einer neuer Report erstellt werden muss. So kann zum Beispiel ein Self-Service Report für die individuelle Auswertung jeder Produktlinie verwendet werden.
\newline \textbf{Testkriterium}: Über eine Benutzeroberfläche können Reports nach spezifischen Anforderungen gefiltert/angepasst werden.
\newline \textbf{Priorität}: 2

\subsection{Automatische Dokumentation des Datenflusses (nach Art. 30 DSGVO)} \label{sec:anforderungsspezifikation:datenflussDokumentation}
\textbf{Beschreibung}: Es soll eine automatische Dokumentation aller Verarbeitungstätigkeiten erfolgen, die die Anforderungen aus Art. 30 DSGVO erfüllt.
\newline \textbf{Begründung}: Der Art. 30 DSGVO erfordert eine detaillierte Dokumentation aller Verarbeitungsschritte an Kundendaten. Dieser Prozess soll zur Zeit- und Kostenersparnis automatisiert werden.
\newline \textbf{Testkriterium}: Für jeden (EU) Kunden liegt eine automatisch erstellte Dokumentation gemäß Art. 30 DSGVO vor.
\newline \textbf{Priorität}: 3

\subsection{Klassifizierung von Daten} \label{sec:anforderungsspezifikation:DatenKlassifizierung}
\textbf{Beschreibung}: Es soll möglich sein, die Daten nach festen Regeln zu klassifizieren. 
\newline \textbf{Begründung}: Die Datenklassifizierung ist notwendig für die Einhaltung des Datenschutzes.
\newline \textbf{Testkriterium}: Allen Daten im BI-System kann eine Klassifizierung zugeordnet werden.
\newline \textbf{Priorität}: 1