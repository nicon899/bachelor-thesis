\begin{scriptsize}
\begin{longtable}{
|p{.7\textwidth   - 2.0\tabcolsep}
|P{.15\textwidth - 2.0\tabcolsep}
|P{.15\textwidth - 2.0\tabcolsep}
|}
\caption[Auswertung des Prototyps]{Überblick über die erfüllten Testkriterien mit dem Prototyp im Vergleich zum on-premise BI-System} \label{table:auswertung} \\

\hline
Anforderungen
& On-premise BI-System 
& Prototyp Cloud BI 
\\ \hline
\endfirsthead

\hline
& On-premise BI-System 
& Prototyp Cloud BI 
\\ \hline
\endhead

%%%%%%%%%%%%%%%%%%%%%%%%%%%%%%%%%%%%%%%%%%%%%%%%%%%%%%%%%%%%%%%%%%%%%%%%%%%%%%%%%%%%%%%%%%%%%%%%%%%%%%%%%%%%%%%%%%%%%%%%%%%%%%%%%%%%%%%%%%%
\textbf{\nameref{sec:anforderungsspezifikation:funktionaleAnforderungen}}
&  % on-premise
&  % cloud
\\ \hline

\nameref{sec:anforderungsspezifikation:datenintegrationOnPremDB}
&  \cmark % on-premise
&  \cmark % cloud
\\ \hline

\nameref{sec:anforderungsspezifikation:datenintegrationCloudDB}
&  \cmark % on-premise
&  \cmark % cloud
\\ \hline

\nameref{sec:anforderungsspezifikation:datenintegrationREST}
&  \cmark % on-premise
&  \cmark % cloud
\\ \hline

\nameref{sec:anforderungsspezifikation:dauerhaftesSpeichern}
&  \cmark % on-premise
&  \cmark % cloud
\\ \hline

\nameref{sec:anforderungsspezifikation:datentransformation}
&  \cmark % on-premise
&  \cmark % cloud
\\ \hline

\nameref{sec:anforderungsspezifikation:datenAuswertung}
&  \cmark % on-premise
&  \cmark % cloud
\\ \hline

\nameref{sec:anforderungsspezifikation:datenanalysePythonUndR}
&  \xmark % on-premise
&  (+)\textsuperscript{1} % cloud
\\ \hline

\nameref{sec:anforderungsspezifikation:reports}
&  \cmark % on-premise
&  \cmark % cloud
\\ \hline

\nameref{sec:anforderungsspezifikation:selfServiceReports}
&  \xmark  % on-premise
&  \cmark % cloud
\\ \hline

\nameref{sec:anforderungsspezifikation:datenflussDokumentation}
&  \xmark % on-premise
&  (\cmark)\textsuperscript{2} % cloud
\\ \hline

\nameref{sec:anforderungsspezifikation:DatenKlassifizierung}
&  \xmark % on-premise
&  (\cmark)\textsuperscript{2} % cloud
\\ \hline

%%%%%%%%%%%%%%%%%%%%%%%%%%%%%%%%%%%%%%%%%%%%%%%%%%%%%%%%%%%%%%%%%%%%%%%%%%%%%%%%%%%%%%%%%%%%%%%%%%%%%%%%%%%%%%%%%%%%%%%%%%%%%%%%%%%%%%%%%%%
\textbf{Nicht-Funktionale Anforderungen}
&
&
\\ \hline

\nameref{sec:anforderungsspezifikation:vielfältigeVisualisierungsmöglichkeiten}
&  \cmark % on-premise
&  \cmark % cloud
\\ \hline

\nameref{sec:anforderungsspezifikation:aktualität}
&  \xmark % on-premise
&  (\cmark)\textsuperscript{3} % cloud
\\ \hline

\nameref{sec:anforderungsspezifikation:schnelleAntwortzeitenDerReports}
& \cmark % on-premise
& \cmark % cloud
\\ \hline

\nameref{sec:anforderungsspezifikation:Datenkonsistenz}
&  \cmark % on-premise
&  \cmark % cloud
\\ \hline

\nameref{sec:anforderungsspezifikation:verfügbarkeit}
&  \cmark % on-premise
&  \cmark % cloud
\\ \hline

\nameref{sec:anforderungsspezifikation:AutomatischeFehlerbehandlung}
&  \cmark % on-premise
&  \cmark % cloud
\\ \hline

\nameref{sec:anforderungsspezifikation:speicherkapazität}
&  \cmark % on-premise
&  \cmark % cloud
\\ \hline

\nameref{sec:anforderungsspezifikation:QuellsystemeÄndern}
&  \cmark % on-premise
&  \cmark % cloud
\\ \hline

\nameref{sec:anforderungsspezifikation:skalierungDerSpeicherkapazität}
&  \xmark % on-premise
&  \xmark % cloud
\\ \hline

\nameref{sec:anforderungsspezifikation:langlebigkeit}
&  \xmark\textsuperscript{4} % on-premise
&  \cmark % cloud
\\\hline

\nameref{sec:anforderungsspezifikation:fehlerquellenIdentifizieren}
&  \cmark % on-premise
&  \cmark % cloud
\\ \hline

\nameref{sec:anforderungsspezifikation:einheitlicheTechnologie}
& \cmark % on-premise
& \xmark % cloud
\\\hline

\nameref{sec:anforderungsspezifikation:rbac}
&  \cmark % on-premise
&  \cmark % cloud
\\ \hline

\nameref{sec:anforderungsspezifikation:SAG_AD}
&  \cmark % on-premise
&  \cmark % cloud
\\ \hline

\nameref{sec:anforderungsspezifikation:zugriffStandort}
&  \cmark % on-premise
&  \cmark % cloud
\\ \hline

\nameref{sec:anforderungsspezifikation:löschenKundendaten}
&  \cmark % on-premise
&  \cmark\textsuperscript{5} % cloud
\\ \hline

\nameref{sec:anforderungsspezifikation:ZugriffMitBetriebsrat}
& \xmark % on-premise
& (-)\textsuperscript{1} % cloud
\\  \hline

\nameref{sec:anforderungsspezifikation:verschlüsselung}
&  \cmark % on-premise
&  \cmark % cloud
\\\hline

\nameref{sec:anforderungsspezifikation:dsgvo}
&  \cmark % on-premise
&  (\cmark)\textsuperscript{6} % cloud
\\ \hline

%%%%%%%%%%%%%%%%%%%%%%%%%%%%%%%%%%%%%%%%%%%%%%%%%%%%%%%%%%%%%%%%%%%%%%%%%%%%%%%%%%%%%%%%%%%%%%%%%%%
\textbf{Summe erfüllter Anforderungen}
&  22 % on-premise
&  22 | (27)\textsuperscript{7} % cloud
\\ \hline

Erfüllte Funktionale Anforderungen
&  7 % on-premise
&  8 | (11)\textsuperscript{7} % cloud
\\

Erfüllte Nicht-Funktionale Anforderungen
&  15 % on-premise
&  14 | (16)\textsuperscript{7} % cloud
\\ \hline
\end{longtable}

\noindent\textsuperscript{1} Voraussichtlich möglich, siehe Abschnitt~\ref{sec:praktischeUmsetzung:ausblick}

\noindent\textsuperscript{2} Teilweise erfüllt, siehe Abschnitt~\ref{sec:praktischeUmsetzung:purviewDatengovernance}

\noindent\textsuperscript{3} Die Performance von Datenintegration und Auswertung ist ausreichend, aber mit Power BI Pro sind nur 8 Aktualisierungen pro Tag möglich, oder es müssen lange Ladezeiten mit Direct Query akzeptiert werden.

\noindent\textsuperscript{4} Eingesetzte Technologien, wie AngularJS, sind bereits veraltet.

\noindent\textsuperscript{5} Durch Purview weniger aufwändig, als beim Bestandssystem (siehe Abschnitt~\ref{sec:praktischeUmsetzung:purviewDatengovernance})

\noindent\textsuperscript{6} Es wurde bei der Konzeption auf die DSGVO-Konformität geachtet. Die Bewertung durch einen Datenschutzbeauftragten ist noch ausstehend.

\noindent\textsuperscript{7} Inklusive teilweiser und voraussichtlich erfüllter Anforderungen. 

\end{scriptsize}