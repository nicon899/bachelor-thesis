\lstset{style=sharpc}
\begin{lstlisting}[frame=single,caption=ETL-Prozess: Aufruf der Funktion,captionpos=b]
public static async Task Run(
    //NCrontab: {second}{minute}{hour}{day}{month}{day-of-week}
    [TimerTrigger("0 0 * * * *")] TimerInfo myTimer, 
    ILogger log, ExecutionContext context)
{
    // Laden des Zugrifftokens für den Zugriff als Managed Identity
    var tokenProvider = new AzureServiceTokenProvider();
    string accessToken = await tokenProvider.
        GetAccessTokenAsync("https://database.windows.net/");

    // Array mit den Namen aller Entitäten 
    string[] entities = new string[] { "region", "company", ... };
    
    // Führe ETL-Prozess für jede Entität aus
    foreach (string entity in entities)
    {
        log.LogInformation($"Start ETL process for entity: {entity}");
        try
        {
            // Laden des letzen Akutalisierungszeitpunkts
            long lastUpdate = getLastUpdateOfEntity(entity, accessToken);
            // Lade SQL-Query aus Datei mit dem Namen der Entität im Verzeichnis SQL
            string sqlPath = Path.Combine(context.FunctionAppDirectory, "sql", $"{entity}.sql");
            string sqlQuery = File.ReadAllText(sqlPath);
            // Extrahieren und Transformieren der Daten aus Quellsystem
            DataTable dataFromSource = extractDataFromSource(entity, sqlQuery, lastUpdate);
            // Laden der Daten in das Zielsystem
            loadDataToDWH(dataFromSource, entity, accessToken);
            log.LogInformation($"Completed ETL for {entity}");
        }
        catch (SqlException ex)
        {
            log.LogError(ex, $"ERROR execution ETL for {entity}; ExceptionMessage: ${ex.ToString()}");
        }
    }
}

public static long getLastUpdateOfEntity(string tableName, string accessToken) {
    DateTime lastUpdateDateTime;
    using (SqlConnection conn = new SqlConnection(Environment.GetEnvironmentVariable("connectionstringDWH")))
    {
        conn.AccessToken = accessToken;
        conn.Open();
        var sqlQuery = $"SELECT MAX(UPDATED) FROM {tableName}";
        using (SqlCommand cmd = new SqlCommand(sqlQuery, conn))
        {
            object lastUpdateObj = cmd.ExecuteScalar();
            if (lastUpdateObj == null || lastUpdateObj == DBNull.Value) { return 0; }
            lastUpdateDateTime = (DateTime)lastUpdateObj;
        }
    }
    TimeSpan diffToOriginTime = lastUpdateDateTime.ToUniversalTime() - DateTime.UnixEpoch;
    long lastUpdate = ((long)diffToOriginTime.TotalSeconds);
    return lastUpdate;
}

  public static DataTable extractDataFromSource(string entity, string sqlQuery, DateTime lastUpdate)
        {
            DataTable dataFromSource = new DataTable();
            using (SqlConnection conn = new SqlConnection(Environment.GetEnvironmentVariable("connectionstringJSM")))
            {
                conn.Open();
                using (SqlCommand cmd = new SqlCommand(sqlQuery, conn))
                {
                    cmd.CommandTimeout = 3600;
                    cmd.Prepare();
                    cmd.Parameters.AddWithValue("@lastUpdate", lastUpdate.ToString());
                    using (SqlDataAdapter sda = new SqlDataAdapter(cmd))
                    {
                        sda.Fill(dataFromSource);
                    }
                }
            }
            return dataFromSource;
        }

        public static void loadDataToDWH(DataTable data, string tableName, string accessToken)
        {
            using (SqlConnection conn = new SqlConnection(Environment.GetEnvironmentVariable("connectionstringDWH")))
            {
                conn.AccessToken = accessToken;
                conn.Open();

                // create temp table
                string sqlCreateTmpTable = $"SELECT TOP 0 * INTO #tmp{tableName} FROM {tableName}";
                using (SqlCommand cmd = new SqlCommand(sqlCreateTmpTable, conn))
                {
                    cmd.CommandTimeout = 3600;
                    cmd.ExecuteNonQuery();
                }

                // bulk copy data into temp table
                using (SqlBulkCopy bulkCopy = new SqlBulkCopy(conn))
                {
                    bulkCopy.DestinationTableName = $"#tmp{tableName}";
                    bulkCopy.WriteToServer(data);
                }

                // update table
                string sqlUpdateTable = $"BEGIN TRANSACTION; DELETE FROM {tableName} WHERE IssueID IN (SELECT IssueID FROM #tmp{tableName}); INSERT INTO {tableName} SELECT * FROM #tmp{tableName}; COMMIT TRANSACTION; DROP TABLE #tmp{tableName}";
                using (SqlCommand cmd = new SqlCommand(sqlUpdateTable, conn))
                {
                    cmd.CommandTimeout = 3600;
                    cmd.ExecuteNonQuery();
                }
            }
        }
    }
\end{lstlisting}