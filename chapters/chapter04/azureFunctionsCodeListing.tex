\lstset{style=sharpc}
\begin{lstlisting}[caption=Quellcode für die Datenintegration mit Azure Functions,captionpos=b,label={lst:azFunc},escapeinside=``]
public static async Task Run(
    // NCrontab: {second}{minute}{hour}{day}{month}{day-of-week}
    [TimerTrigger("0 0 * * * *")] TimerInfo myTimer,
    ILogger log, // ILogger-Objekt sendet zu protokollierende Nachrichten an Application Insights
    ExecutionContext context) {
    // Laden des Zugrifftokens für den Zugriff als Managed Identity
    var tokenProvider = new AzureServiceTokenProvider();
    string accessToken = await tokenProvider.
        GetAccessTokenAsync("https://database.windows.net/");

    // Array mit den Namen aller Entitäten 
    string[] entities = new string[] { "region", "company", ... };
    
    // Führe ETL-Prozess für jede Entität aus
    foreach (string entity in entities) {
        try {
            // Abfragen des letzten Aktualisierungszeitpunkts
            long lastUpdate = getLastUpdateOfEntity(entity, 
                                accessToken);
            // Lade SQL-Query aus Datei mit Namen der Entität
            string sqlPath = Path.Combine(
                                context.FunctionAppDirectory, "sql", 
                                $"{entity}.sql");
            string sqlQuery = File.ReadAllText(sqlPath);
            // Extrahieren und Transformieren der Daten aus Quellsystem in Zwischenspeicher(DataTable)
            DataTable dataFromSource = 
                extractDataFromSource(entity,sqlQuery,lastUpdate);
            // Laden der Daten in das Zielsystem
            loadDataToDWH(dataFromSource, entity, accessToken);
            // Erfolgreiche Durchführung wird protokolliert
            log.LogInformation($"Completed ETL for {entity}");
        } catch (Exception ex) {
            // Aufgetretene Fehler werden protokolliert
            log.LogError(ex, $"ERROR execution ETL for {entity} -
                ExceptionMessage: ${ex.ToString()}");
        }
    }
}
`\newpage`
// Gibt den letzten Aktualisierungszeitpunkt einer Tabelle in Unixzeit zurück
public static long getLastUpdateOfEntity(string tableName, 
    string accessToken) {
    DateTime lastUpdateDateTime;
    // Verbindung zu Zielsystem mit Verbindungszeichenfolge aus der Umgebungsvariable "connectionstringDWH"
    using (SqlConnection conn = new SqlConnection(Environment.GetEnvironmentVariable("connectionstringDWH"))) {
        conn.AccessToken = accessToken;
        conn.Open();
        // Abfrage des letzten Aktualisierungszeitpunkts mit SQL
        var sqlQuery = $"SELECT MAX(UPDATED) FROM {tableName}";
        using (SqlCommand cmd = new SqlCommand(sqlQuery, conn))
        {
            object lastUpdateObj = cmd.ExecuteScalar();
            // Gibt den Wert 0 zurück, falls Zieltabelle leer ist
            if (lastUpdateObj == null || lastUpdateObj 
                == DBNull.Value) { return 0; }
            lastUpdateDateTime = (DateTime) lastUpdateObj;
        }
    }
    // Umwandlung von DateTime-Objekt zu Unixzeit
    TimeSpan diffToUnixEpoch = lastUpdateDateTime.ToUniversalTime() - DateTime.UnixEpoch;
    long lastUpdate = ((long) diffToUnixEpoch.TotalSeconds);
    return lastUpdate;
}

// Extrahiert und Transformiert Daten aus Quellsystem
 public static DataTable extractDataFromSource(string entity, string sqlQuery, DateTime lastUpdate) {
    // DataTable-Objekt zum zwischenspeichern der Datensätze
    DataTable dataFromSource = new DataTable();
    // Verbindung zu Quellsystem mit Verbindungszeichenfolge aus der Umgebungsvariable "connectionstringJSM"
    using (SqlConnection conn = new SqlConnection(Environment.GetEnvironmentVariable("connectionstringJSM"))) {
        conn.Open();
        // Führe übergebene SQL-Query aus
        using (SqlCommand cmd = new SqlCommand(sqlQuery, conn)) {
            // Berücksichtigung des letzten Aktualisierungszeitpunkts durch Parameter
            cmd.Prepare();
            cmd.Parameters.AddWithValue("@lastUpdate", lastUpdate.ToString());
            using (SqlDataAdapter sda = new SqlDataAdapter(cmd))
            {
                // Schreibe SQL-Ergebnis in DataTable-Objekt
                sda.Fill(dataFromSource);
            }
        }
    }
    return dataFromSource;
}
`\newpage`
// Lädt zwischengespeicherte Daten in DWH
public static void loadDataToDWH(
    DataTable data, // Daten die in Zielsystem geladen werden
    string tableName, // Name der Zieltabelle
    string accessToken) {
    // Verbindung zu Zielsystem
    using (SqlConnection conn = new SqlConnection(Environment.GetEnvironmentVariable("connectionstringDWH"))) {
        conn.AccessToken = accessToken;
        conn.Open();
        // Beginne Datenbanktransaktion
        using (SqlTransaction transaction=conn.BeginTransaction()) {
            // Erstelle Temporäre Tabelle mit gleichen Spalten wie Zieltabelle
            string sqlCreateTmpTable = $"SELECT TOP 0 * INTO #tmp{tableName} FROM {tableName}";
            using (SqlCommand cmd = new SqlCommand(sqlCreateTmpTable, conn, transaction)) {
                cmd.ExecuteNonQuery();
            }
            
            // Einfügen der extrahierten Daten in temporäre Tabelle in einem Massenkopiervorgang
            using (SqlBulkCopy bulkCopy = new SqlBulkCopy(conn, SqlBulkCopyOptions.Default, transaction))
            {
                bulkCopy.DestinationTableName = $"#tmp{tableName}";
                bulkCopy.WriteToServer(data);
            }
            
            // Aktualisieren der Zieltabelle mit temporärer Tabelle, die anschließend gelöscht wird
            string sqlUpdateTable = 
                $"DELETE FROM {tableName} WHERE ID IN (SELECT ID FROM #tmp{tableName});"
                + $"INSERT INTO {tableName} SELECT * FROM #tmp{tableName};"
                + $"DROP TABLE #tmp{tableName};";
            using (SqlCommand cmd = new SqlCommand(sqlUpdateTable, conn, transaction)) {
                cmd.ExecuteNonQuery();
            }
            
            // Ausführen der Datenbanktransaktion und Rollback im Fehlerfall
            try {
                transaction.Commit();
            } catch (Exception ex) {
                transaction.Rollback();
                throw (ex);
            }
        }
    }
}
\end{lstlisting}