\section{Auswertung} \label{sec:umsetzung:auswertung}
Mit Hilfe des Prototyps wird die neue BI-Architektur in der Praxis getestet. Zu den Testfällen gehört das Erstellen von Reports für die integrierten Quellsysteme. Die Reports sollen vergleichbar zu den aus dem on-premise BI-System sein. Es sollen berechnete Kennzahlen, Tabellen und Diagramme, die den zeitlichen Verlauf eines Werts darstellen, angezeigt werden. Durch das Validieren der Inhalte können gleichzeitig mehrere Anforderungen überprüft werden. Denn Fehler bei der Datenintegration oder der Auswertung würden auch zu fehlerhaften Reports führen.

% Beim aktuellen BI-System wird für das Reporting ein auf Javascript-Bibliotheken aufbauendes Webframework verwendet. Wird vom Management ein neuer Report gewünscht, muss dieser implementiert und anschließend als Webseite veröffentlicht werden. Daraus folgt die Schwierigkeit, dass Reports nur mit Grundkenntnissen in Webentwicklung erstellt werden können. Mit \textit{Power BI} soll es zukünftig einer deutlich größeren Personengruppe möglich sein, benutzerdefinierte Reports zu erstellen.