\section{Auswertung} \label{sec:umsetzung:auswertung}
Mit Hilfe des Prototyps konnte die neue BI-Architektur in der Praxis erprobt werden. Der in \textit{Power BI} erstellte Report kann als End-to-End-Test verstanden werden. Denn Fehler bei der Datenintegration, Speicherung oder Auswertung würden auch zu fehlerhaften Reports führen. Deswegen wurde der Inhalt der Reports, nachdem die Datenintegration durchgeführt wurde, validiert und entsprach den Erwartungen. Daraus folgt, dass die Grundfunktionen eines BI-Systems erfolgreich umgesetzt wurden.

Um den Prototyp bezüglich aller Anforderungen zu bewerten, wurden zusätzlich die zu Beginn definierten Testfälle (\hyperref[sec:anforderungsspezifikation:funktionaleAnforderungen]{siehe Anforderungsspezifikation}) ausgeführt. Ein Großteil davon wurde bereits durch die Umsetzung in Abschnitt~\ref{sec:praktischeUmsetzung:Migration} abgedeckt. Das Ergebnis und ein Vergleich zum on-premise BI-System kann aus Tabelle~\ref{table:auswertung} entnommen werden. 

\begin{scriptsize}
\begin{longtable}{
|p{.7\textwidth   - 2.0\tabcolsep}
|P{.15\textwidth - 2.0\tabcolsep}
|P{.15\textwidth - 2.0\tabcolsep}
|}
\caption[Auswertung]{Auswertung} \label{table:auswertung} \\

\hline
Anforderungen
& On-premise BI-System 
& Prototyp Cloud BI 
\\ \hline
\endfirsthead

\hline
& On-premise BI-System 
& Prototyp Cloud BI 
\\ \hline
\endhead

%%%%%%%%%%%%%%%%%%%%%%%%%%%%%%%%%%%%%%%%%%%%%%%%%%%%%%%%%%%%%%%%%%%%%%%%%%%%%%%%%%%%%%%%%%%%%%%%%%%%%%%%%%%%%%%%%%%%%%%%%%%%%%%%%%%%%%%%%%%
\textbf{\nameref{sec:anforderungsspezifikation:funktionaleAnforderungen}}
&  % on-premise
&  % cloud
\\ \hline

\nameref{sec:anforderungsspezifikation:datenintegrationOnPremDB}
&  \cmark % on-premise
&  \cmark % cloud
\\ \hline

\nameref{sec:anforderungsspezifikation:datenintegrationCloudDB}
&  \cmark % on-premise
&  \cmark % cloud
\\ \hline

\nameref{sec:anforderungsspezifikation:datenintegrationREST}
&  \cmark % on-premise
&  \cmark % cloud
\\ \hline

\nameref{sec:anforderungsspezifikation:dauerhaftesSpeichern}
&  \cmark % on-premise
&  \cmark % cloud
\\ \hline

\nameref{sec:anforderungsspezifikation:datentransformation}
&  \cmark % on-premise
&  \cmark % cloud
\\ \hline

\nameref{sec:anforderungsspezifikation:datenAuswertung}
&  \cmark % on-premise
&  \cmark % cloud
\\ \hline

\nameref{sec:anforderungsspezifikation:datenanalysePythonUndR}
&  \xmark % on-premise
&  ?\textsuperscript{1} % cloud
\\ \hline

\nameref{sec:anforderungsspezifikation:reports}
&  \cmark % on-premise
&  \cmark % cloud
\\ \hline

\nameref{sec:anforderungsspezifikation:selfServiceReports}
&  \xmark  % on-premise
&  \cmark % cloud
\\ \hline

\nameref{sec:anforderungsspezifikation:datenflussDokumentation}
&  \xmark % on-premise
&  (\cmark)\textsuperscript{2} % cloud
\\ \hline

\nameref{sec:anforderungsspezifikation:DatenKlassifizierung}
&  \xmark % on-premise
&  (\cmark)\textsuperscript{2} % cloud
\\ \hline

%%%%%%%%%%%%%%%%%%%%%%%%%%%%%%%%%%%%%%%%%%%%%%%%%%%%%%%%%%%%%%%%%%%%%%%%%%%%%%%%%%%%%%%%%%%%%%%%%%%%%%%%%%%%%%%%%%%%%%%%%%%%%%%%%%%%%%%%%%%
\textbf{Nicht-Funktionale Anforderungen}
&
&
\\ \hline

\nameref{sec:anforderungsspezifikation:vielfältigeVisualisierungsmöglichkeiten}
&  \cmark % on-premise
&  \cmark % cloud
\\ \hline

\nameref{sec:anforderungsspezifikation:aktualität}
&  \xmark % on-premise
&  (\cmark)\textsuperscript{3} % cloud
\\ \hline

\nameref{sec:anforderungsspezifikation:schnelleAntwortzeitenDerReports}
& \cmark % on-premise
& \cmark % cloud
\\ \hline

\nameref{sec:anforderungsspezifikation:Datenkonsistenz}
&  \cmark % on-premise
&  \cmark % cloud
\\ \hline

\nameref{sec:anforderungsspezifikation:verfügbarkeit}
&  \cmark % on-premise
&  \cmark % cloud
\\ \hline

\nameref{sec:anforderungsspezifikation:AutomatischeFehlerbehandlung}
&  \cmark % on-premise
&  \cmark % cloud
\\ \hline

\nameref{sec:anforderungsspezifikation:speicherkapazität}
&  \cmark % on-premise
&  \cmark % cloud
\\ \hline

\nameref{sec:anforderungsspezifikation:QuellsystemeÄndern}
&  \cmark % on-premise
&  \cmark % cloud
\\ \hline

\nameref{sec:anforderungsspezifikation:skalierungDerSpeicherkapazität}
&  \xmark % on-premise
&  \xmark % cloud
\\ \hline

\nameref{sec:anforderungsspezifikation:langlebigkeit}
&  \cmark % on-premise
&  \cmark % cloud
\\\hline

\nameref{sec:anforderungsspezifikation:fehlerquellenIdentifizieren}
&  \cmark % on-premise
&  \cmark % cloud
\\ \hline

\nameref{sec:anforderungsspezifikation:einheitlicheTechnologie}
& \cmark % on-premise
& \xmark % cloud
\\\hline

\nameref{sec:anforderungsspezifikation:rbac}
&  \cmark % on-premise
&  \cmark % cloud
\\ \hline

\nameref{sec:anforderungsspezifikation:SAG_AD}
&  \cmark % on-premise
&  \cmark % cloud
\\ \hline

\nameref{sec:anforderungsspezifikation:zugriffStandort}
&  \cmark % on-premise
&  \cmark % cloud
\\ \hline

\nameref{sec:anforderungsspezifikation:löschenKundendaten}
&  \cmark % on-premise
&  \cmark\textsuperscript{4} % cloud
\\ \hline

\nameref{sec:anforderungsspezifikation:ZugriffMitBetriebsrat}
& \xmark % on-premise
& ?\textsuperscript{1} % cloud
\\  \hline

\nameref{sec:anforderungsspezifikation:verschlüsselung}
&  \cmark % on-premise
&  \cmark % cloud
\\\hline

\nameref{sec:anforderungsspezifikation:dsgvo}
&  \cmark % on-premise
&  (\cmark)\textsuperscript{5} % cloud
\\ \hline
\end{longtable}

\noindent\textsuperscript{1} Voraussichtlich möglich, siehe Abschnitt~\ref{sec:praktischeUmsetzung:ausblick}

\noindent\textsuperscript{2} Teilweise erfüllt, siehe Abschnitt~\ref{sec:praktischeUmsetzung:purviewDatengovernance}

\noindent\textsuperscript{3} Die Performance von Datenintegration und Auswertung ist ausreichend, aber mit Power BI Pro sind nur 8 Aktualisierungen pro Tag möglich, oder es müssen lange Ladezeiten mit Direct Query akzeptiert werden.

\noindent\textsuperscript{4} Durch Purview weniger aufwändig, als beim Bestandssystem (siehe Abschnitt~\ref{sec:praktischeUmsetzung:purviewDatengovernance})

\noindent\textsuperscript{5} Es wurde bei der Konzeption auf die DSGVO-Konformität geachtet. Die Genehmigung durch einen Datenschutzbeauftragten konnte aus zeitlichen Gründen noch nicht erfolgen.

\end{scriptsize}

Insgesamt ist festzustellen, dass der Prototyp die Anforderungen besser erfüllt als das on-premise BI-System, abgesehen von einer Ausnahme: \textit{\nameref{sec:anforderungsspezifikation:einheitlicheTechnologie}}. Aufgrund der vielen verschiedenen Ressourcen ist der Prototyp komplexer als das on-premise BI-System und verwendet mehr unterschiedliche Technologien. Daneben konnte nur die Anforderung an die automatische Skalierung der Speicherkapazität nicht erfüllt werden. Das wird jedoch als unkritisch angesehen, weil der Skalierungsprozess vollständig von Azure übernommen wird und dieser nur manuell gestartet werden muss \cite{reagan_web_2018}. Obwohl \textit{Purview} die Erwartungen nicht vollständig erfüllen konnte, bietet es trotzdem einen Mehrwert, indem es Aufgaben (teil-)automatisiert, die aktuell manuell umgesetzt werden. Dadurch könnte zukünftig die Einhaltung von Gesetzen, wie der DSGVO erleichtert werden. Abgesehen davon bietet die entworfene Cloud BI-Architektur neue Funktionen und Möglichkeiten. Beim on-premise BI-System wird für das Reporting eine auf Javascript-Bibliotheken aufbauendes Webframework verwendet. Wenn ein neuer Report bereitgestellt werden soll, muss dieser implementiert und anschließend als Webseite veröffentlicht werden. Daraus folgt die Schwierigkeit, dass Reports nur mit Kenntnissen in Webentwicklung erstellt werden können. \textit{Power BI} könnte es zukünftig einer deutlich größeren Personengruppe ermöglichen, benutzerdefinierte Reports zu erstellen. Ein weiterer Vorteil der Cloud Lösung ist, dass Auswertungen mit Machine Learning, zukünftig ohne grundlegende Veränderungen am System, ergänzt werden können. Bei den nicht-funktionalen Anforderungen ist besonders die verbesserte Aktualität und die längere Lebensdauer, weil keine veralteten Technologien verwendet werden, hervorzuheben.