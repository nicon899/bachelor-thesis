\section{Bereitstellen der Infrastruktur} \label{sec:umsetzen:bereitstellenInfrastruktur}

\subsection{Zentraler Datenspeicher} \label{subsec:umsetzen:bereitstellenInfrastruktur:datenspeicher}
Als zentraler Datenspeicher wird die \textit{Azure SQL Database} verwendet. Für diese stehen zwei Preismodelle zur Auswahl. Das schon länger existierende, basiert auf DTUs (database transaction units), welche eine Kombination von Rechen-, Speicher- und I/O Ressourcen darstellen. Hierbei ist das Problem, dass keine unabhängige Skalierung möglich ist, falls zum Beispiel der Speicherbedarf überdurchschnittlich ist, aber kaum Rechenleistung benötigt wird. Das vCore-basierte Modell dagegen, ermöglicht diese unabhängige Skalierung und wird von Microsoft empfohlen. Aus diesen beiden Gründen soll hier das vCore-basierte Modell verwendet werden. Damit stehen drei Preisstufen zur Auswahl, wobei es sich im Zweifel empfiehlt zu Beginn die niedrigste zu wählen, da bei Bedarf jederzeit auf eine höhere umgestiegen werden kann. Für die niedrigste Stufe \textit{General Purpose} kann zwischen den Optionen \textit{Provisioned} und \textit{Serverless} für die Rechenressourcen gewählt werden. Bei \textit{Provisioned} werden die Rechenressourcen dauerhaft zur Verfügung gestellt. Das eignet sich besonders bei regelmäßiger Nutzung mit höherer Durchschnittsauslastung. Die serverlose Alternative skaliert die Ressourcen automatisch bei Bedarf und ist sinnvoll, wenn die Nutzung unvorhersehbar und unregelmäßig, dafür insgesamt jedoch geringer ist. Bei der Annahme, dass die Datenbank stündlich aktualisiert wird und anschließend die neuen Daten fürs Reporting gelesen werden, ist die erste Option voraussichtlich günstiger und wird deswegen gewählt.

Als Collation wird die Variante \textit{SQL{\_}Latin1{\_}General{\_}CP1{\_}CI{\_}AS} festgelegt. Dieses spezifiziert neben dem Alphabet, das zum Sortieren verwendet werden soll, dass Akzente, jedoch nicht Groß- und Kleinschreibung, berücksichtigt werden. \cite[vgl.][]{mauri_azure_2021}

Neben der Datenbank muss ebenfalls ein Datenbankserver erstellt und angegeben werden. Hierbei handelt es sich nicht um eine SQL-Server Instanz, sondern nur um einen logischen Server mit Metadaten zur Verwaltung einer oder mehrerer \textit{Azure SQL-Datenbanken}. Für den Server wird ein Name und die Region angeben. Die Region des logischen Servers wird auch für alle zugehörigen Datenbanken übernommen \cite{ward_azure_2021}. Für eine höhere Sicherheit wird hier außerdem festgelegt, dass die Authentifizierung für den Datenbankzugriff nur über das \ac{aad} erfolgen kann und es wird ein Nutzer aus dem \ac{aad} als Administrator angegeben.