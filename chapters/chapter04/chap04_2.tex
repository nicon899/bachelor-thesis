\subsection{Erläuterung der Ressourcenkonfigurationen} \label{subsec:infra:konfig}

\subsubsection{Netzwerk} \label{subsec:infra:konfig:netzwerk}
\textit{...}

\subsubsection{Zentraler Datenspeicher} \label{subsec:infra:konfig:datenspeicher}
Als zentraler Datenspeicher wird die \textit{Azure SQL Database} verwendet. Für diese stehen zwei Preismodelle zur Auswahl. Das anfänglich einzige, basiert auf DTUs (database transaction units), welche eine Kombination von Rechen-, Speicher- und I/O Ressourcen darstellen. Hierbei ist das Problem, dass keine unabhängige Skalierung möglich ist, falls zum Beispiel der Speicherbedarf überdurchschnittlich ist, aber kaum Rechenleistung benötigt wird. Das vCore-basierte Modell dagegen, ermöglicht diese unabhängige Skalierung und wird von Microsoft empfohlen. Aus diesen beiden Gründen soll hier das vCore-basierte Modell verwendet werden. Damit stehen drei Preisstufen zur Auswahl, wobei es sich im Zweifel empfiehlt zu Beginn die niedrigste zu wählen, da bei Bedarf jederzeit auf eine höhere umgestiegen werden kann. Für die niedrigste Stufe \textit{General Purpose} kann zwischen den Optionen \textit{Provisioned} und \textit{Serverless} für die Rechenressourcen gewählt werden. Bei \textit{Provisioned} werden die Rechenressourcen dauerhaft zur Verfügung gestellt. Das eignet sich besonders bei regelmäßiger Nutzung mit höherer Durchschnittsauslastung. Die serverlose Alternative skaliert die Ressourcen automatisch bei Bedarf und ist sinnvoll, wenn die Nutzung unvorhersehbar und unregelmäßig, dafür insgesamt jedoch geringer ist. Bei der Annahme, dass die Datenbank stündlich aktualisiert wird und anschließend die neuen Daten fürs Reporting gelesen werden, ist die erste Option voraussichtlich günstiger und wird deswegen gewählt. Als Collation wird die Variante \textit{SQL{\_}Latin1{\_}General{\_}CP1{\_}CI{\_}AS} festgelegt. Dieses spezifiziert neben dem Alphabet, das zum Sortieren verwendet werden soll, dass Akzente, jedoch nicht Groß- und Kleinschreibung, berücksichtigt werden. \cite[vgl.][]{mauri_azure_2021}

Neben der Datenbank muss ebenfalls ein Datenbankserver erstellt und angegeben werden. Hierbei handelt es sich nicht um eine SQL-Server Instanz, sondern nur um einen logischen Server mit Metadaten zur Verwaltung einer oder mehrerer \textit{Azure SQL-Datenbanken}. Für den Server wird ein Name und die Region angeben. Die Region des logischen Servers wird auch für alle zugehörigen Datenbanken übernommen. Für eine höhere Sicherheit wird hier außerdem festgelegt, dass die Authentifizierung für den Datenbankzugriff nur über das \ac{aad} erfolgen kann und es wird eine Admin-Gruppe aus dem \ac{aad} als Datenbankadministrator angegeben. Damit \textit{Azure Functions} später auf die Datenbank zugreifen kann, wird die dazugehörige statische IP-Adresse in den Firewall-Einstellungen erlaubt \cite[vgl.][]{ward_azure_2021}.

\subsubsection{Datenintegration} \label{subsec:infra:konfig:functions}
Damit \textit{Azure Functions} für den ETL-Prozess genutzt werden kann, wird eine Funktionsapp, als Container für die Funktionen erstellt. Für diese kann aus drei verschiedene Plänen gewählt werden. Im \textit{Verbrauchsplan} besteht keine Kontrolle über die verwendete Infrastruktur, sondern der Dienst skaliert automatisch nach Bedarf. Dadurch kann es zwischen Aufruf und Ausführung einer Funktion zu einer Verzögerung kommen. Die Kosten sind abhängig von der Ausführungsdauer der Funktionen. Beim \textit{Premiumplan} wird ebenfalls automatisch skaliert, es kann jedoch entsprechend der Leistungsanforderungen die eingesetzte Rechenleistung gewählt werden. Die Ressourcen sind bei diesem Plan jederzeit betriebsbereit, wodurch die Ausführung von Funktionen direkt bei Aufruf ohne Verzögerung beginnt. Der \textit{Dedizierte Plan} teilt der Funktionsapp feste Rechenressourcen zu, welche manuell oder automatisch skaliert werden können. Nur im Premium und dedizierten Plan ist es möglich ein virtuelles Netzwerk zu verwenden, weswegen der Verbrauchsplan hier nicht verwendet werden kann. Da die ETL-Prozesse nicht häufig ausgeführt werden, aber die Ausführungszeit vorrauschichtlich länger dauern kann, ist der \textit{Dedizierte Plan} hier die beste Option. Die Runtime legt fest, in welcher Umgebung die Funktionen ausgeführt werden und damit auch die zu verwendende Programmiersprache. Da in Azure die \textit{C\# .NET} Runtime, laut einer von IEEE veröffentlichten Studie, die performanteste Option ist \cite{jackson_investigation_2018}, soll diese für den ETL-Prozess verwendet werden. Des Weiteren wird ein \textit{Azure Storage Account} vorausgesetzt, in dem unter anderem der Quellcode der Funktionen gespeichert wird. Die Firewall wird so konfiguriert, dass alles außer die Funktionsapp blockiert wird. Außerdem wird nur die Verbindung mit verschlüsselten Protokolle wie HTTPS und FTPS zugelassen. Der Dienst \textit{Application Insights} für das Monitoring erfordert keine spezielle Konfiguration und muss lediglich mit der Funktionsapp verknüpft werden. \cite[vgl.][]{satapathi_hands-azure_2021}

Für den ausgehenden Datenverkehr soll eine statische IP-Adresse verwendet werden. Dafür wird ein Subnetz im virtuellen Netzwerk erstellt und verknüpft. In diesem Subnetz wird der Datenverkehr über ein NAT Gateway mit einer eigenen öffentlichen IP-Adresse weitergeleitet. Zusätzlich wird eine Funktionskonfiguration hinzugefügt, welche erzwingt, dass alle Funktionen ihren ausgehenden Datenverkehr über dieses NAT Gateway leiten \cite[vgl.][]{msdoc_22_func_ip}. Für die Verbindung zu dem Quellsystem auf einem on-premise Server wird ein Relay erstellt. Dem Relay wird die Konfiguration der Hybridverbindung hinzugefügt, welche die Netzwerkadresse und den Port des lokalen Endpunkts definiert. Abschließend wird die Hybridverbindung mit der Funktionsapp verknüpft. Nachdem die Hybridverbindung in Azure bereitgestellt wurde, kann eine dazugehörige Gateway-Verbindungszeichenfolge entnommen werden. Damit erfolgt die manuelle Installation und Einrichtung der Anwendung \textit{Hybrid Connection Manager} auf dem on-premise Server \cite[vgl.][]{msdoc_22_func_hybridConn}.

\subsubsection{Zugriff auf Datenbank im virtuellen Netzwerk} \label{subsec:infra:konfig:vm}
Mit einer Windows \ac{vm}, auf der zwei Anwendungen installiert werden, soll es den Diensten \textit{Power BI} und \textit{Purview} ermöglicht werden, auf die Daten in der \textit{SQL Database} zuzugreifen, ohne dass die Firewall für alle öffentlichen Adressen geöffnet werden muss. 

Das Betriebssystem und die Daten der \ac{vm} werden auf einer virtuellen \textit{Premium SSD} Festplatte gespeichert. Diese befindet sich einem lokal redundanten Speichercontainer und wird dort, zum Schutz vor Verlust, mindestens dreimal repliziert. Daten, die sich in Ruhe befinden, werden auf der virtuellen Festplatte automatisch verschlüsselt. Jedoch gewährleistet Microsoft für eine einzelne \ac{vm}-Instanz nicht die angeforderte Verfügbarkeit von 99,95\%. Aus diesem Grund wird eine zweite \ac{vm}-Instanz in einer anderen Verfügbarkeitszone erstellt. Die Azure Region besteht aus drei getrennten Verfügbarkeitszonen, welche voneinander unabhängige Hardware verwenden. Dadurch ist sichergestellt, dass beim Ausfall eines Datencenters immernoch eine weitere Instanz aktiv ist und die \ac{sla} liegt bei einer Verfügbarkeit von 99,99\% \cite[vgl.][]{soh_data_2020}.

Um die Sicherheit zu optimieren sollen die \acp{vm} keine öffentliche IP-Adresse haben, damit der Zugriff nur innerhalb des virtuellen Netzwerks möglich ist. Der Dienst Bastion bietet für diesen Anwendungsfall, eine komfortable und sichere Möglichkeit über das Azure Portal im Webbrowser eine RDP-Verbindung mit der \acp{vm} herzustellen. Bastion muss sich dazu in einem Subnet innerhalb des gleichen virtuellen Netzwerks wie die \acp{vm} befinden \cite[vgl.][]{herath_azure_2022}.

% Bevor mit dem Erstellen des sogenannten \textit{Scans} begonnen werden kann, muss die Anwendung \textit{Self hosted-integration runtime} auf den beiden \acp{vm} installiert werden.

\subsubsection{Reporting} \label{subsec:infra:konfig:powerbi}
\textit{Power BI} stellt gegenüber den anderen hier aufgeführten Diensten eine Ausnahme dar, weil die Bereitstellung nicht direkt über Azure erfolgen soll. Denn für  \textit{Power BI} existiert ein eigenes Lizenzmodell mit monatlichen Fixbeträgen, die in diesem Fall günstiger als die nutzungsbasierte Abrechnung in Azure sind. Die Lizenzen können im \textit{Office 365 Admin Center} gekauft und zugeordnet werden. Damit kann anschließend auf von Microsoft bereitgestellte Ressourcen zugegriffen werden \cite[vgl.][]{gunnarsson_pro_2020}.

Es stehen die Lizenzen \textit{Free}, \textit{Pro} und \textit{Premium} zur Auswahl. Die Free Lizenz eignet sich nur für die persönliche Verwendung, da Reports nicht mit anderen Personen geteilt werden können und ist daher für das BI-System ungenügend. Erst Power BI Pro ermöglicht das kollaborative Arbeiten mit Reports. Die Pro-Lizenz ist Teil vom \textit{Office 365 Enterprise E5} Abonnent und steht damit allen potenziellen Endnutzern bereits zur Verfügung. Bei \textit{Premium} handelt es sich dagegen nicht um einzelne Lizenzen für die Angestellten, sondern um eine für das gesamte Unternehmen. Dabei ist der größte Unterschied gegenüber \textit{Pro}, dass dedizierte Hardware bereitgestellt wird und damit kein Ressourcenpool mit anderen Kunden geteilt wird. Personen, die in Premium Reports erstellen oder bearbeiten wollen, müssen zusätzlich über eine Pro-Lizenz verfügen. Für Betrachter genügt jedoch die Free Lizenz. Weitere nennenswerte Vorteile sind die höhere Kapazitätsgrenze von Datensets und die Möglichkeit diese häufiger zu aktualisieren \cite[vgl.][]{gunnarsson_pro_2020}. An dieser Stelle ist es schwierig zu beurteilen, ob die Leistung der Pro Lizenz ausreichend ist, oder die Premium Variante benötigt wird. Zum Sparen von Kosten soll zunächst versucht werden, mit dem bereits vorhandenen \textit{Power BI Pro} auszukommen. Beim Testen des Prototyps soll dann validiert werden, ob damit alle Anforderungen erfüllt werden können.

\subsubsection{Datengovernance} \label{subsec:infra:konfig:purview}
Für \textit{Azure Purview} sind vergleichsweise wenig Entscheidungen notwendig. ...

\subsubsection{Sichere Aufbewahrung von Schlüsseln} \label{subsec:infra:konfig:keyVault}
Für die gespeicherten Passwörter im \textit{Azure Key Vault} kann zwischen Hardware- und Softwaremodus entschieden werden. Bei beiden erfolgt die Speicherung auf einem Hardware-Sicherheitsmodul, jedoch werden nur beim Hardwaremodus alle Operationen, wie Ver- und Entschlüsselung, ebenfalls in diesen Modul durchgeführt. Damit verspricht der Hardwaremodus eine höhere Sicherheit, setzt jedoch die Premium-Variante des Key Vaults voraus. In der Premium-Variante stehen außerdem stärkere Schlüssel zur Verfügung \cite{haunts_key_2019}. Aus diesen Gründen wird hier die Premium, gegenüber der Standard~=Variante bevorzugt, da sie mehr Möglichkeiten, mit denen die Sicherheit weiter erhöht werden kann, bietet.

Des Weiteren wird die Verwendung von \ac{rbac} festgelegt. Alternativ würde ein eigenes Zugriffsrichtliniensystem des \textit{Key Vaults} verwendet werden. In den Firewall Regeln werden Anfragen aus nicht explizit zugelassen Netzwerken blockiert. Zu den Ausnahmen zählt die statische IP-Adresse der Funktionsapp und das virtuelle Netzwerk, in dem sich die \acp{vm} befinden \cite[vgl.][]{herath_azure_2022}.

\subsubsection{Vollständige Infrastruktur} \label{subsec:infra:konfig:VollständigeInfrastruktur}