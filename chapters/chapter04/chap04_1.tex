\section{Bereitstellen der Infrastruktur} \label{sec:infra}

\subsection{Vorgehen und Vorbereitung} \label{subsec:infra:azDevOps}
Zu Beginn wird über das Azure Portal einmalig eine Ressourcengruppe erstellt. Dieser Container für zusammengehörige Ressourcen erlaubt das Teilen eines gemeinsamen Lebenszyklus, von Berechtigungen und Richtlinien. Dabei wird die Region West-Europa als Standort gewählt, die für alles Ressourcen in diesem Prototyp verwendet werden soll. 

\subsubsection{Berechtigungen und Rollen} \label{subsec:infra:prep:aad}
% ServicePrincipal: purview, devOps
Die Managed Identities der Funktionsapp und der für \textit{Azure Purview} erstellte Dienstprinzipal bekommen eine Rolle zugewiesen, die es erlaubt, Passwörter aus dem \textit{Azure Key Vault} zu lesen. Der Dienstprinzipal für \textit{Purview} erhält Leserechte und
die Managed Identity von \textit{Azure Functions} bekommt für den ETL-Prozess Lese- und Schreibrechte auf der Datenbank.

\subsubsection{Festlegen von Richtlinien mit Azure Policy} \label{subsec:infra:prep:policy}
Mit \textit{Azure Policy} können Richtlinien festgelegt werden, die die Einhaltung von Unternehmensstandards absichern. So kann zum Beispiel das Hinzufügen oder Verändern von Ressourcen verhindern, wenn dadurch gegen Compliance-Anforderungen verstoßen werden würde. Sie schützen somit die Infrastruktur gleichzeitig vor böswilligen als auch ungewollt vorgenommen Änderungen. Mehrere Richtlinien können zu einer Initiative gruppiert werden und damit gemeinsam einem Geltungsbereich zugewiesen werden \cite{de_tender_azure_2019}.

Während es für ein produktives System sinnvoll sein kann, für jede Konfiguration, die einen Einfluss auf die Compliance hat, eine eigene Richtlinie zu erstellen, wird für den Prototyp nur eine Initiative mit zwei enthaltenen Richtlinie zu Testzwecken erstellt. Für die Erste soll eine vorgefertigte Richtlinie verwendet werden, welche erzwingt, dass die Verschlüsselung der ruhenden Daten der SQL-Datenbank aktiviert ist. Als Zweites wird eine selbstdefinierte Richtlinie verwendet, die sicherstellt, dass sich alle Ressourcen in einem europäischen Rechenzentrum befinden. Damit wird möglichst einfach verhindert, dass sich Daten in einer nicht DSGVO-konformen Region befinden.

\subsubsection{DevOps mit Infrastructure-As-Code} \label{subsec:infra:konfig:policy}
Alle benötigten Ressourcen können über einen Webbrowser im Azure Portal erstellt und verwaltet werden \cite{chilberto_building_2020}. Zum Erstellen einer komplexeren Architektur ist jedoch der \ac{arm} besser geeignet. Damit können beliebige Azure Komponenten flexibel über parametrisierbare Templates im JSON-Format, erstellt und verwaltet werden. Das macht \ac{arm} weniger fehleranfällig und zeitsparender als die Konfiguration aller einzelnen Komponenten über das Azure Portal \cite{monadjemi_azure-administration_2017}. Aus den genannten Gründen soll ein ARM-Template erstellt werden, um das Cloud-BI System bereitzustellen. Zusätzlich ermöglicht der \textit{Infrastructure-As-Code} Ansatz die Anwendung von DevOps. Das bedeutet, dass die ARM-Templates in einer Versionskontrolle verwaltet werden und die Infrastruktur über Pipelines automatisiert bereitgestellt werden kann. Dadurch kann auch die Verwaltung von mehreren Umgebungen (üblicherweise Entwicklung, Test und Produktion) vereinfacht werden \cite{riscutia_data_2021}. Für den DevOps Prozess wird die Plattform GitHub verwendet. Hier wird ein GitHub Actions Workflow erstellt, der automatisch ausgeführt wird, wenn ein neuer commit gepusht wird und dann die Bereitstellung der Ressourcen übernimmt. Der Workflow wird in einer YAML-Datei definiert. Der Inhalt wurde aus \citetitle{rendon_deploy_2022} übernommen und besteht aus zwei Aktionen. Zunächst wird ein Login ausgeführt, bevor anschließend das ARM-Template in Azure bereitgestellt wird. Damit diese beiden Aktionen möglich sind wird in Azure ein Dienstprinzipal erstellt, der die notwendigen Rechte hat, um Ressourcen zu verändern. Der Dienstprinzipal wird in GitHub als verschlüsselte Umgebungsvariable hinterlegt. Das gilt auch für weitere Parameter, wie den Namen der Ressourcengruppe oder geheimzuhaltende Parameter wie Passwörter \cite[vgl.][]{rendon_deploy_2022}.

Da das \ac{arm}-Template umfangreich ist und im Prinzip nur eine Auflistung von Definitionen und Parametern der Ressourcen ist, wird es in dieser Arbeit nicht aufgeführt. Stadtessen wird die Infrastruktur, die mit dem \ac{arm}-Template bereitgestellt wird, im Rahmen des nächsten Abschnitts erläutert.