\section{Vorgehen und Vorbereitung} \label{sec:intro:azDevOps}
Alle benötigten Ressourcen können über einen Webbrowser im Azure Portal erstellt und verwaltet werden \cite{chilberto_building_2020}. Zum Erstellen einer komplexeren Architektur ist jedoch der \ac{arm} besser geeignet. Damit können beliebige Azure Komponenten zu einer logischen Einheit (Ressourcengruppe) zusammengefasst werden. Eine Ressourcengruppe kann mit ARM, flexibel über parametrisierbare Templates im JSON-Format, erstellt und verwaltet werden. Das macht ARM weniger fehleranfällig und zeitsparender als die Konfiguration aller einzelnen Komponenten über das Azure Portal \cite{monadjemi_azure-administration_2017}. Aus den genannten Gründen soll ein ARM-Template erstellt werden, um das Cloud-BI-System bereitzustellen.

Zusätzlich ermöglicht der \textit{Infrastructure-As-Code} Ansatz die Anwendung von DevOps. Das bedeutet, dass die ARM-Templates in einer Versionskontrolle verwaltet werden und die Infrastruktur über Pipelines automatisiert bereitgestellt werden kann. Dadurch kann auch die Verwaltung von mehreren Umgebungen (üblicherweise Entwicklung, Test und Produktion) vereinfacht werden \cite{riscutia_data_2021}.
