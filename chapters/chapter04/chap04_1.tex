\section{Vorgehen und Vorbereitung} \label{sec:intro:azDevOps}
Alle benötigten Ressourcen können über einen Webbrowser im Azure Portal erstellt und verwaltet werden \cite{chilberto_building_2020}. Zum Erstellen einer komplexeren Architektur ist jedoch der \ac{arm} besser geeignet. Damit können beliebige Azure Komponenten zu einer logischen Einheit (Ressourcengruppe) zusammengefasst werden. Eine Ressourcengruppe kann mit \ac{arm}, flexibel über parametrisierbare Templates im JSON-Format, erstellt und verwaltet werden. Das macht \ac{arm} weniger fehleranfällig und zeitsparender als die Konfiguration aller einzelnen Komponenten über das Azure Portal \cite{monadjemi_azure-administration_2017}. Aus den genannten Gründen soll ein ARM-Template erstellt werden, um das Cloud-BI-System bereitzustellen. Zusätzlich ermöglicht der \textit{Infrastructure-As-Code} Ansatz die Anwendung von DevOps. Das bedeutet, dass die ARM-Templates in einer Versionskontrolle verwaltet werden und die Infrastruktur über Pipelines automatisiert bereitgestellt werden kann. Dadurch kann auch die Verwaltung von mehreren Umgebungen (üblicherweise Entwicklung, Test und Produktion) vereinfacht werden \cite{riscutia_data_2021}. Für den DevOps Prozess wird die Plattform GitHub verwendet. Hier wird ein GitHub Actions Workflow erstellt, der automatisch ausgeführt wird, wenn ein neuer commit gepusht wird und dann die Bereitstellung der Ressourcen übernimmt. 

Zu Beginn wird in Azure einmalig eine Ressourcengruppe erstellt. Diese dient als Container für zusammengehörige Ressourcen und erlaubt das Teilen von Lebenszyklus, Berechtigungen und Richtlinien. Dabei wird die Region West-Europa als Standort gewählt, der für alles Ressourcen in diesem Prototyp verwendet werden soll. Außerdem wird ein Dienstprinzipal erstellt, der die notwendigen Rechte hat, um Ressourcen in der Gruppe zu verändern. Der Dienstprinzipal wird in GitHub als verschlüsselte Umgebungsvariable hinterlegt. Das gilt auch für weitere Parameter, wie den Namen der Ressourcengruppe oder geheimzuhaltende Passwörter. Der Workflow wird in einer YAML-Datei definiert. Der Inhalt wurde aus \citetitle{rendon_deploy_2022} übernommen und besteht aus zwei Aktionen. Zunächst wird ein Login ausgeführt, bevor anschließend das ARM-Template in Azure bereitgestellt wird \cite[vgl.][]{rendon_deploy_2022}.