\chapter{Einleitung} \label{ch:intro}
Die Strategie der Software AG sieht es vor, zukünftig alle internen Anwendungen in der Azure Cloud zu betreiben. Daraus folgt, dass die Anwendungen, die aktuelle auf eigener Hardware betrieben werden in die Cloud umgezogen oder mit einer Alternative ersetzt werden müssen. Nachdem dies bereits für die ersten Anwendungen erfolgt ist, beschäftigt sich diese Arbeit mit der Frage, wie ein on-premise BI-System nach Azure migriert werden kann.

\section{Motivation} \label{sec:intro:motivation}
Für Unternehmen hat es einen hohen Stellenwert, Entscheidungen auf Basis von aktuellen, korrekten und qualitativ hochwertigen Informationen treffen zu können. Denn so kann das Risiko für Fehlentscheidungen minimiert werden \cite{grunwald_business_2009}. Mit einem BI-System können wertvolle Informationen aus operativen Daten gewonnen werden \cite{muller_business_2013}. Zu den zahlreichen Gründen für einen Umstieg auf eine Cloud BI-Lösung gehört die Skalierbarkeit und der geringere Wartungsaufwand, da weder Betrieb und Wartung von eigener Hardware noch das Installieren und Aktualisieren von Software notwendig ist. Außerdem wird es bei Bedarf durch die nahezu unbegrenzt hochskalierbare Rechenleistung in der Cloud ermöglicht, große Datenmengen effizient zu analysieren und in Reports zu verwenden, ohne dass dafür unverhältnismäßige Kosten entstehen \cite{gurjar_cloud_2013}. Denn bei hohem Rechenbedarf können jederzeit zusätzliche Ressourcen hinzugeschaltet werden. Werden diese nicht mehr benötigt, können sie abbestellt werden und müssen nicht mehr bezahlt werden. Im Vergleich zu einem System, das immer auf die maximale Auslastung vorbereitet ist, kann dadurch das Kosten/Nutzen-Verhältnis optimiert werden \cite{ouf_cloud_2011}.

\section{Ziel der Arbeit} \label{sec:intro:ziele}
Ziel dieser Arbeit ist ein Konzept für den Umstieg vom vorhandenen on-premise BI-System zu einer modernen Alternative in der Azure Cloud. Das Konzept soll aus einer Cloud BI-Architektur und der Beschreibung eines möglichen Migrationsprozesses bestehen. Neben der Funktionalität sollen besonders Sicherheit und Datenschutz beim Entwurf der Architektur im Fokus stehen. Durch eine prototypische Durchführung des Umstiegs, soll bestätigt werden, dass das Konzept funktioniert und den Anforderungen in der Praxis genügt. Des Weiteren soll durch den Vergleich mit dem on-premise BI-System eine Entscheidungsgrundlage geboten werden, ob die Migration zu der vorgestellten Lösung empfehlenswert wäre.

\section{Gliederung} \label{sec:intro:struktur}
Zu Beginn werden die wichtigsten Grundlagen für die Migration des on-premise BI-Systems in die Cloud erläutert. Darunter der Aufbau des Bestandssystems, \ac{bi} in Azure, die Migrationsstrategie und die Spezifikation der Anforderungen. Das folgende Kapitel beschäftigt sich mit dem Entwurf einer Cloud BI-Architektur. Dazu werden verschiedene Azure Dienste vorgestellt und evaluiert. Darauf basierend wird eine vollständige Architektur für das Cloud BI-System vorgestellt. Im nächsten Kapitel wird ein prototypischer Umstieg zu der vorgestellten BI-Lösung beschrieben. In diesem Zusammenhang wird die Cloud BI-Lösung basierend auf den Anforderungen bewertet und anschließend mit dem Bestandssystem verglichen. Danach wird anhand bestehender Literatur und der neu gewonnenen Erkenntnisse diskutiert, wie sinnvoll der tatsächliche Umstieg von on-premise zu Cloud-BI wäre und was dabei zu beachten ist. Abschließend werden die Ergebnisse zusammengefasst und es wird ein Ausblick gegeben, wie diese zukünftig genutzt werden können.