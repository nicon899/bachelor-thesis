\section{Migrationsstrategie} \label{sec:beschreibungMigrationsstrategie}
In der Literatur wird zwischen drei Migrationsstrategien unterschieden \cite{juan-verdejo_moving_2014}:
\begin{enumerate}
\item \textbf{Teilweise nach Fachbereichen}: Die Migration wird nur für einzelne Fachbereiche durchgeführt. Zum Beispiel wird das BI-System für den Kundensupport migriert, während das BI-System für den Vertrieb weiterhin on-premise betrieben wird.
\item \textbf{Teilweise nach Schichten}: Es werden nicht alle funktionalen Schichten in die Cloud migriert. Aus Sicherheitsbedenken könnte beispielsweise die Speicherung von Daten weiterhin on-premise erfolgen, während die Datenanalysen und Auswertungen mit Cloud Ressourcen durchgeführt werden.
\item \textbf{Vollständig}: Das gesamte on~=premise System wird durch eine Cloud~=Lösung ersetzt.
\end{enumerate}
Die 1. Option wäre hier nicht sinnvoll, da dazu das on-premise BI-System zunächst nach Fachbereichen zerlegt werden müsste. Aktuell werden zum Beispiel die gleichen Reports von mehreren Fachbereichen genutzt und die Auswertung wird nur durch einen Filter angepasst. \textit{2.} wird hauptsächlich aus Sicherheitsgründen angewendet, führt dabei aber oft zu Performance Einbußen. Da die Sicherheit bereits ein wichtiges Kriterium der neuen BI~=Architektur ist und die Rahmenbedingungen die Verwendung von Azure vorsehen, wird diese Strategie als unpassend zu der Anforderungsspezifikation erachtet. Daraus folgt, dass für das Bestandssystem die vollständige Migrationsstrategie angewendet werden soll.