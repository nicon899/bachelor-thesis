\section{Migrationsstrategie} \label{sec:beschreibungMigrationsstrategie}
In der Literatur wird zwischen drei Migrationsstrategien unterschieden \cite{juan-verdejo_moving_2014}:
\begin{enumerate}
\item \textbf{Teilweise nach Fachbereichen}: Die Migration wird nur für einzelne Fachbereiche durchgeführt. Zum Beispiel wird das BI-System für den Kundensupport migriert, während das BI-System für den Vertrieb weiterhin on-premise betrieben wird.
\item \textbf{Teilweise nach Schichten}: Es werden nicht alle funktionalen Schichten in die Cloud migriert. Aus Sicherheitsbedenken könnte beispielsweise die Speicherung von Daten weiterhin on-premise erfolgen, während die Datenanalysen und Auswertungen mit Cloud Ressourcen durchgeführt werden.
\item \textbf{Vollständig}: Das gesamte on~=premise BI~=System wird durch eine Cloud~=Lösung ersetzt.
\end{enumerate}
Für das bestehende BI-System soll die vollständige Migrationsstrategie angewendet werden. \textit{1.} wäre nicht sinnvoll, da für das gesamte BI-System nur ein Team zuständig ist. \textit{2.} wird hauptsächlich aus Sicherheitsgründen angewendet, führt dabei aber oft zu Performance Einbußen. Da die Sicherheit bereits ein wichtiges Kriterium der neuen BI~=Architektur ist, wird diese Strategie als nicht notwendig erachtet.