\section{Microsoft Azure Cloud} \label{sec:grundlagen:bi_in_der_cloud_mit_azure}
Die Microsoft Azure Cloud wird intern bereits für andere Anwendungsfälle genutzt und ist die einzige Cloud Plattform, die aktuell von der Software AG für die Speicherung und Verarbeitung von Unternehmensdaten zugelassen ist. Mit Microsoft bestehen bereits entsprechende Verträge und die notwendige Zustimmung vom Betriebsrat liegt vor. Die Verwendung eines anderen Cloud~=Providers wäre theoretisch möglich, ist aber in der aktuellen Unternehmensstrategie nicht vorgesehen und würde damit einen deutlich größeren organisatorischen Aufwand bedeuten. Deswegen beschränkt sich diese Arbeit auf Microsoft Azure.

Neben den internen Gründen gibt es aber auch fachliche Argumente für die Wahl von Azure. Hier werden alle Vorteile der Cloud mit gleichzeitig hoher Flexibilität geboten. Die große Auswahl an unterstützten Betriebssystemen, Plattformen und Tools soll es den Kunden ermöglichen, nahezu alle gewünschten Technologien zu verwenden. Des Weiteren hat Microsoft auf der ganzen Welt verteilt Datenzentren und ist auf eine schnelle Disaster Recovery vorbereitet \cite{modi_azure_2020}. Für das hier verfolgt Ziel ist jedoch folgende Aussage aus \citetitle{modi_azure_2020} am wichtigsten "\textit{[...] it provides a host of interconnected services that can pass data among themselves. With such capabilities in place, data can be processed to generate meaningful knowledge and insights}" \cite{modi_azure_2020}. Die Azure Cloud bietet demnach einige Möglichkeiten für die Konstruktion einer BI~=Infrastruktur. Dafür relevante Grundlagen werden im Folgenden beschrieben.

\subsection{Business Intelligence in Azure} \label{subsec:grundlagen:azure:bi}
In der Cloud wird eine Vielzahl an Ressourcen, wie virtualisierte Hardware oder Dienste, zur Verfügung gestellt. Diese können dynamisch konfiguriert und skaliert werden, sodass immer die benötigte Leistung zur Verfügung steht. Microsoft Azure ist eine Sammlung an solchen Ressourcen, die sich aktuell in 22 Kategorien einteilen lassen. Manche Dienste können dabei zu mehreren Kategorien gehören. Für den Entwurf der neuen BI~=Architektur werden Dienste aus den folgenden Kategorien in Betracht gezogen \cite{chilberto_building_2020}:
\begin{itemize}
\item \textbf{Integrations}~=Dienste können eingesetzt werden, um Daten aus on~=premise oder Cloud Quellsystemen zu extrahieren.
\item \textbf{Datenbank}~=Dienste sind unter anderem relationale Datenbanken, wie Azure SQL, MySQL und PostgreSQL. Eine Alternative dazu ist die NoSQL~=Datenbank Azure Cosmos DB, welche mehrere Datenmodelle unterstützt.
\item \textbf{Speicher}~=Dienste sind in verschiedenen Formen und Ausprägungen vorhanden und können nach individuellen Anforderungen ausgewählt werden. Das ermöglicht es, eine große Menge an Daten, die sowohl strukturiert als auch unstrukturiert sein können, möglichst günstig abzuspeichern.
\item \textbf{Analyse}~=Dienste helfen, die benötigten Erkenntnisse aus den Daten zu gewinnen.
\item \textbf{KI und Machine Learning} ermöglicht komplexere Analysen. Zum Beispiel können, basierend auf historischen Daten, Vorhersagen über die Zukunft getroffen werden. Diese können beispielsweise genutzt werden, um die zukünftige Entwicklung von KPIs abzuschätzen.
\item \textbf{Identitäts}~=Dienste werden genutzt, um sicherzustellen, dass nur berechtigte Nutzer Zugriff auf Anwendungen und Daten erhalten.
\item \textbf{Netzwerk}~=Dienste werden verwendet, um eine (sichere) Verbindung zwischen den verschiedenen Ressourcen herzustellen.
\item \textbf{Sicherheit} ist ein wichtiges Kriterium für Cloud~=Systeme. Azure stellt unter anderem Dienste zum Erkennen von Sicherheitsrisiken und zur Überwachung bereit.
\item \textbf{Verwaltung und Governance} umfasst Dienste zur Automatisierung, sowie zur Verwaltung und Überwachung der Cloud~=Ressourcen. Dazu gehört das Erstellen von Backups, sowie das Monitoring von Anwendungen, der Infrastruktur und den dadurch entstehenden Kosten.
\item \textbf{Migrations}~=Dienste sollen beim Umstieg von on~=premise zu Cloud Ressourcen unterstützen. Sie könnten zum Beispiel eingesetzt werden, um Daten aus dem bestehenden \ac{dwh} in eine Cloud Alternative zu übertragen.
\end{itemize}
Beim Vergleich der Azure Kategorien mit den funktionalen Schichten des on~=premise BI~=Systems decken die \textit{Integrations~=Dienste} die Funktionalität der \textit{Datenintegration} ab. Das \ac{dwh} kann in Azure entweder mit den \textit{Datenbank~=}, den \textit{Speicher~=Diensten} oder einer Kombination von beiden ersetzt werden. Die \textit{Analyse~=Dienste} entsprechen der \textit{auswertungsorientierten Schicht} und den \textit{BI~=Anwendungen} des on~=premise Systems. Zusätzlich neue Auswertungsmöglichkeiten bieten die Dienste aus \textit{KI und Machine Learning}. Die Kategorien \textit{Identität}, \textit{Netzwerk}, \textit{Sicherheit} und \textit{Verwaltung und Governance} sind vergleichbar mit dem \textit{Warehouse Management}.

\subsection{Dienste für Sicherheit und Datenschutz} \label{subsec:grundlagen:azure:sicherheitUndDatenschutz}
Neben den Azure Diensten zur Erfüllung der funktionalen Anforderungen, die erst in Abschnitt~\ref{sec:konzeption:evaAuswertung} ausgewählt werden, sollen weitere Dienste für die Sicherheit und den Datenschutz beim Entwurf der neuen BI~=Architektur berücksichtigt werden.

\subsubsection{Azure Policy} \label{subsec:grundlagen:azure:sicherheitUndDatenschutz:ap}
Mit \textit{Azure Policy} kann die Einhaltung bestimmter Compliance Anforderungen für alle Azure Ressourcen erzwungen werden. Dies wird für das System regelmäßig überprüft. Dazu werden zunächst die einzuhaltenden Parameter, wie beispielsweise der Serverstandort, festgelegt. So kann der Versuch eine Ressource zu erstellen, die gegen die festgelegten Compliance Standards verstößt, verhindert werden. Durch \textit{Azure Policy} kann also eine konsistente Einhaltung der Compliance im gesamten BI~=System sichergestellt werden \cite{stefanovic_azure_2021}.

\subsubsection{Azure Key Vault} \label{subsec:grundlagen:azure:sicherheitUndDatenschutz:keyVault}
Der \textit{Azure Key Vault} ist ein sicherer Cloud~=Speicherplatz für Schlüssel, Passwörter und Zertifikate. Dafür verfügt jedes Rechenzentrum über ein Hardware~=Sicherheitsmodul, welches durch den Einsatz von verschiedenen Sensoren sowohl Hacking~=Attacken als auch physische Zugriffsversuchen erkennen kann. In diesem Fall werden alle Passwörter automatisch von der Festplatte gelöscht. Deswegen liegt immer mindestens eine Kopie der Daten in einem anderen Rechenzentrum vor \cite{haunts_key_2019}.

\subsubsection{Azure Active Directory} \label{subsec:grundlagen:azure:sicherheitUndDatenschutz:aad}
\ac{aad} ist ein cloudbasierter Identitäts- und Zugangsverwaltungsdienst. Durch diesen können sich die Nutzer mit Benutzernamen und Passwort authentifizieren. Für eine erhöhte Sicherheit wird außerdem die Verwendung der inzwischen Standard gewordenen \ac{mfa} unterstützt.

Das \ac{aad} ermöglicht die Umsetzung von \ac{rbac}. Mit \ac{rbac} können die Zugriffsrechte einer Identität auf die Azure Ressourcen granular festgelegt werden \cite{stefanovic_azure_2021}. Eine Identität kann dabei ein Benutzer, eine Gruppe, ein service principal oder eine managed identity sein. Service principals können Azure Diensten zugewiesen werden, um diesen Zugriff auf andere Ressourcen zu gewähren. Eine managed identiy erfüllt den gleichen Zweck, jedoch müssen damit keine Zugangsdaten verwaltet werden, da dies vom \ac{aad} übernommen wird \cite{copeland_security_2021}.

%\subsubsection{Microsoft Defender for Cloud} \label{subsec:grundlagen:azure:sicherheitUndDatenschutz:asc}
% Der \textit{Microsoft Defender for Cloud} (ehemals Azure Security Center) soll einen Überblick über die Sicherheit der gesamten Infrastruktur geben. Dazu gibt es unter anderem ein Sicherheitsrating, das als Maßstab für die Bewertung der Systemsicherheit verwendet werden kann  \cite{buchanan_azure_2022}.