\section{Spezifikation der Anforderungen}
Die Anforderungen an die neue BI-Architektur werden in iterativen Meetings mit den Stakeholdern festgelegt. Eine besondere Herausforderung hierbei ist es, dass die Anforderungsanalyse unvoreingenommen durchgeführt wird. Die Anforderungen sollten weder aus der aktuellen BI-Lösung noch aus individuellen Lösungsideen abgeleitet werden. Stattdessen sollten sie sich auf die Essenz des Systems beziehen, also den geschäftlichen Grund für den Einsatz des BI-Systems \cite[vgl.][]{robertson_mastering_2013}.

Als Grundlage für die Anforderungsspezifikation wird eine aktuelle Version des \citetitle{robertson_volere_2020}\textit{s} \cite{robertson_volere_2020} verwendet. Dieses gibt eine grundlegende Struktur und Kategorisierung vor. Hier wird außerdem Wert daraufgelegt, dass beim Beschreiben einer Anforderung ein Testkriterium festgelegt wird. Dadurch werden die Anforderungen messbar und es kann später eindeutig festgestellt werden, ob sie erfüllt wurden. Abgesehen davon werden jedoch nur die Bestandteile der Vorlage übernommen, die für die Evaluation in Abschnitt~\ref{sec:evaluation} als relevant angesehen werden.

Weggelassen wurden Anforderungen an die Benutzerfreundlichkeit und kulturelle Anforderungen, weil für den Endnutzer Reports die einzige Schnittstelle zum BI-System sind. In dieser Arbeit steht jedoch die Architektur im Vordergrund, deswegen ist nur die grundsätzliche Möglichkeit, vielfältige Reports zu erstellen, wichtig. Es ist hier nicht vorgesehen, die Reports zu optimieren. Auch operative und umweltbezogene Anforderungen sind für die Evaluation nicht relevant, da kein Release geplant ist und alle Azure Dienste in einer vergleichbaren physikalischen Umgebung laufen.