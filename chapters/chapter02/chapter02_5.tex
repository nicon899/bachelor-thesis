\section{Verwandte Arbeiten} \label{ch:verwandteArbeiten}
Bereits 2011 wurden von \citeauthor{ouf_cloud_2011} die Vorteile von Cloud gegenüber on-premise BI beschrieben und eine Architektur vorgeschlagen. Die Beschreibung ist allerdings kurz und abstrakt gehalten. Es wird weder auf das Vorgehen beim Entwurf noch auf eine konkrete Umsetzung eingegangen \cite[vgl.][]{ouf_cloud_2011}.

Ein ähnliches Vorgehen wie hier kann in \citetitle{oliver_norkus_rabic_2016} gefunden werden. Die Auswahl der Dienste erfolgt basierend auf Anforderungen und ob diese wirklich erfüllt werden, wird mit einem Prototyp validiert. Im Unterschied zu dieser Arbeit wurde eine Cloud Plattform von \textit{SAP} gewählt und die Beschreibung ist deutlich abstrakter \cite[vgl.][]{oliver_norkus_rabic_2016}.

In \citetitle{borosch_cloud_2021} wird die Umsetzung einer Referenzarchitektur aus Azure Diensten detailliert behandelt. Diese unterscheidet sich durch andere Anforderungen und damit den verwendeten Diensten. Außerdem wird beim Entwerfen der Architektur anders vorgegangen, als in dieser Arbeit. Dort wird schrittweise durch das Hinzufügen weiterer Dienste, eine immer komplexere und mächtiger werdende Architektur entworfen. Die Auswahl der verwendeten Dienste wird allerdings nur mit der Funktionalität begründet und es werden keine Alternativen berücksichtigt \cite[vgl.][]{borosch_cloud_2021}.

Nachdem die wichtigsten Begriffe und Grundlagen vorgestellt wurden, wird in den nächsten zwei Kapiteln ein Cloud-BI System entworfen und umgesetzt. Hervorzuheben ist dabei die Auswahl der Azure Dienste, da für die einzelnen Aufgaben nicht immer die naheliegendste Lösung gewählt wird, sondern verschiedene Alternativen betrachtet und in Bezug zu der Anforderungsspezifikation bewertet werden. Eine weitere Besonderheit ist, dass in Kapitel~\ref{ch:praktischeUmsetzung} nicht nur die Umsetzung eines Prototyps beschrieben wird, sondern auch darauf eingegangen wird, wie eine Migration mit möglichst geringen Aufwand durchgeführt werden kann.