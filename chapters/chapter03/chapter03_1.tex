\section{Anforderungen}
\label{sec:anforderungen}
Im Folgenden soll ein kurzer Überblick über die wichtigsten Anforderungen gegeben werden. Eine vollständige Anforderungsspezifikation kann in Anhang~\ref{ch:anforderungsspezifikation} gefunden werden. 

\subsection{Funktionale Anforderungen}
Es sollen Daten aus verschiedenen on-premise und Cloud Quellsystemen integriert werden können und die extrahierten Daten müssen langfristig gespeichert werden. Auf den gespeicherten Daten sollen automatisierte Analysen und Auswertungen durchgeführt werden können. Die Ergebnisse der Auswertungen sollen visualisiert in Reports angezeigt werden. Neben den fest definierten Auswertungen sollen auch Self-Service-Reports bereitgestellt werden. Außerdem soll die Möglichkeit bestehen, das BI-System um fortgeschrittene Analysen, wie Machine Learning, zu erweitern, ohne dass grundlegende Veränderungen an der Architektur vorgenommen werden müssen. Die gespeicherten Daten sollen nach festen Regeln automatisch klassifiziert werden, da dies beeinflusst wer Zugriff auf die Daten haben darf. Laut DSGVO Art. 30 ist es erforderlich ein \textit{Verzeichnis von Verarbeitungstätigkeiten} zu führen. Diese Dokumentation aller Übertragungen und Verarbeitungsschritte der Daten soll automatisiert werden. 

\subsection{Nicht-funktionale Anforderungen}
Zu den wichtigsten Anforderungen an Cloud-Systeme gehören die Sicherheit und der Datenschutz \cite{gurjar_cloud_2013}. Daher soll das neue System bestmöglich vor Angriffen geschützt sein. Daneben soll das in DSGVO Art. 17 geforderte \textit{Recht auf Vergessenwerden}, effizient ausgeführt werden können. Eine zuverlässige Authentifizierung und Autorisierung der Nutzer ist ebenfalls unumgänglich. Für alle Daten und Informationen muss sichergestellt werden, dass nur berechtigte Personen diese sehen können. Worauf ein Mitarbeiter Zugriff haben darf, ist abhängig von seiner Rolle im Unternehmen und seinem regionalen Standort. Im Fehlerfall soll ein zuständiges Team benachrichtigt werden und die aufgetretenen Fehler sollen durch eine Protokollierung leicht nachvollziehbar sein. Ein weiteres Kriterium ist die Verwendung von möglichst wenig unterschiedlichen Technologien. Dadurch soll der Wartungsaufwand und die Anzahl der hierfür notwendigen Personen minimiert werden.