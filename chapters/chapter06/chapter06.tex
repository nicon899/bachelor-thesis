\chapter{Fazit} \label{ch:zusammenfassung}
Diese Arbeit hat sich mit dem strategischen Ziel, ein on-premise BI-System in die Azure Cloud zu migrieren, beschäftigt. Obwohl das Thema dieser Arbeit eine Migration ist, wurde die neue Architektur unabhängig von diesen Vorhaben entworfen. Erst anschließend wurde ein Konzept für den Migrationsprozess entworfen, das möglichst wenig Aufwand verursacht. Durch dieses Vorgehen konnte eine Lösung geschaffen werden, die die Anforderungen besser erfüllt als das on-premise BI-System.

Bei der präsentierten Lösung sind drei Herausforderungen besonders hervorzuheben: \textbf{Erstens} wurden die zu verwendenden Azure Dienste ausgewählt. Offensichtlich konnte dabei nicht jede Möglichkeit berücksichtigt werden, was insbesondere Infrastruktur-Ressourcen, wie eine \ac{vm}, betrifft, mit denen Anwendungen betrieben werden könnten, die nicht von Azure angeboten werden. Unabhängig davon wurde trotzdem ein breites Spektrum an Möglichkeiten abgedeckt. Davon wurden nicht einfach beliebige Dienste gewählt, die eine benötigte Funktionalität bieten, sondern es wurden verschiedene Optionen miteinander verglichen, um eine sinnvolle Auswahl zu treffen. \textbf{Zweitens} wurde eine Architektur entworfen, was, solange nur die funktionalen Anforderungen berücksichtigt werden, trivial ist. Da die Sicherheit und der Datenschutz zentrale Aspekte darstellen, musste die Architektur um weitere Ressourcen ergänzt werden. Besonders die Kommunikation zwischen den BI-Komponenten, mit gleichzeitigem Schutz durch Firewall-Regeln, erfordert ein komplexeres Gesamtsystem. \textbf{Drittens} wurde für den prototypischen Umstieg ein Vorgehen erarbeitet, mit dem die einzelnen Bestandteile des on-premise BI-Systems zu den Azure Diensten migriert werden können. Die Datenbankmigration ist dabei am einfachsten, weil geeignete Tools existieren. Für den \ac{etl}-Prozess mit Datenbank Quellsystemen wurde eine Funktion entwickelt, die es ermöglicht, vorhandene SQL-Skripte mit geringer Anpassung zu übernehmen. Aktuell ist die Funktion auf ein vereinfachtes Datenmodell ausgelegt und muss für das vollständige Datenmodell entweder weiter abstrahiert oder in ähnlicher Form kopiert werden, damit alle Entitäten berücksichtigt werden können. Für die Reports kann der Aufbau und die Darstellungsart von Auswertungen übernommen werden. Alles Weitere muss von Grund auf implementiert werden, was akzeptabel ist, da die Migration ein einmaliger Prozess ist und auch neue Funktionen, wie Self-Service-Reporting und automatisierte Datengovernance, umgesetzt werden.

Abschließend konnte durch den Prototyp bestätigt werden, dass das Cloud BI-System dem Bestandssystem aus Anforderungssicht überlegen ist. Deswegen wäre der einzige Grund, mit der Migration zu warten, die Tatsache, dass sich noch nicht mit den Kosten der neuen Lösung beschäftigt wurde. Es ist damit noch offen, ob durch den Umstieg die Kosten für das BI-System sinken oder steigen würden. Unter dem Gesichtspunkt, dass der Umzug aller internen Anwendungen nach Azure bereits vorgesehen ist, kann die vorgestellte Architektur trotzdem empfohlen werden. Denn für die Komponenten und deren Konfiguration wurde jeweils eine möglichst kosteneffektive Option gewählt, sofern dies aus Literatur oder Dokumentation hergleitet werden konnte. Zusätzlich konnte gezeigt werden, dass das vorgestellte Konzept in der Praxis funktioniert. 

\section{Ausblick}
Die Ergänzung von \ac{aml} wurde bereits beim Entwurf der BI-Architektur vorgesehen. Daher ist es naheliegend, dass dies zukünftig umgesetzt wird, um anschließend das Reporting um fortgeschrittene Analysen zu erweitern. Über die Anforderungen hinaus bietet es sich an zukünftig weitere Möglichkeiten der Azure Cloud zu betrachten. \textit{...}

Eines der größten Schwächen der vorgestellten Lösung ist die Notwendigkeit mindestens zwei \acp{vm} in Azure zu bereitzustellen. Zukünftig wird diese Notwendigkeit voraussichtlich wegfallen, da sowohl \textit{Power BI} als auch \textit{Purview} in einer Vorschauphase die Möglichkeit bieten, direkt auf eine Datenquelle in einem virtuellen Netzwerk zuzugreifen. Wenn diese Funktionen veröffentlicht werden, ist eine Optimierung der Cloud BI-Architektur möglich.