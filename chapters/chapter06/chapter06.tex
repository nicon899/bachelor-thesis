\chapter{Fazit} \label{ch:zusammenfassung}
Diese Arbeit hat sich mit dem Ziel, ein on-premise BI-System in die Azure Cloud zu migrieren, beschäftigt. Die neue BI-Architektur wurde dabei angepasst an die betrieblichen Anforderungen und unabhängig vom Bestandssystem entworfen. Durch dieses Vorgehen konnte eine kosteneffektive Lösung geschaffen werden, die die Anforderungen besser erfüllt als das vorhandene BI-System.

Bei der präsentierten Lösung sind drei Herausforderungen besonders hervorzuheben: \textbf{Erstens} wurden die zu verwendenden Azure Dienste ausgewählt, indem die verschiedenen Optionen zum Erfüllen der Aufgaben des BI-Systems miteinander verglichen wurden. Damit die Evaluation in einem vertretbaren Umfang durchgeführt werden konnte, wurde eine Vorauswahl aus dem breiten Angebot von Azure Diensten mithilfe einer Literaturrecherche getroffen. \textbf{Zweitens} wurde eine Architektur entworfen, die über die offensichtlichen funktionalen Anforderungen hinausgeht und weitere zentrale Aspekte wie Sicherheit und Datenschutz berücksichtigt. Besonders die Kommunikation zwischen den BI-Komponenten, mit gleichzeitigem Schutz durch Firewall-Regeln, erfordert ein komplexeres Gesamtsystem. \textbf{Drittens} wurde für den prototypischen Umstieg ein Vorgehen erarbeitet, mit dem die einzelnen Bestandteile des on-premise BI-Systems zu den Azure Diensten migriert werden können. Für die Datenbankmigration kann auf bereits vorhandene Tools zurückgegriffen werden. Im Rahmen dieser Arbeit wurde für den \ac{etl}-Prozess mit Datenbank Quellsystemen eine Funktion entwickelt, die es ermöglicht, vorhandene SQL-Skripte mit geringer Anpassung zu übernehmen. Aktuell ist die Funktion auf ein vereinfachtes Datenmodell ausgelegt und muss für das vollständige Datenmodell entweder weiter abstrahiert oder in ähnlicher Form kopiert werden, damit alle Entitäten berücksichtigt werden können. Für die Reports kann der Aufbau und die Darstellungsart von Auswertungen übernommen werden. Alles Weitere muss von Grund auf implementiert werden, was akzeptabel ist, da die Migration ein einmaliger Prozess ist und auch neue Funktionen, wie Self-Service-Reporting und automatisierte Datengovernance, umgesetzt werden.

Abschließend konnte durch den Prototyp bestätigt werden, dass das Cloud BI-System dem Bestandssystem aus Anforderungssicht überlegen ist. Die Klärung des Kostenaspekts würde wesentlich weitergehende Untersuchungen erfordern, sodass aktuell keine Aussage getroffen werden kann, ob durch den Umstieg die Kosten für das BI-System sinken oder steigen würden. Unter dem Gesichtspunkt, dass der Umzug der internen Anwendungen nach Azure strategisch gewünscht ist, kann die vorgestellte Architektur trotzdem empfohlen werden, denn für die Komponenten und deren Konfiguration wurde jeweils eine möglichst kosteneffektive Option gewählt, sofern dies aus Literatur oder Dokumentation hergleitet werden konnte. Zusätzlich konnte gezeigt werden, dass das vorgestellte Konzept in der Praxis funktioniert.

\section{Ausblick}
Auf Grundlage der hier vorgestellten Ergebnisse kann zukünftig entschieden werden, ob die vorschlagende Migration durchgeführt werden soll. Wenn das der Fall ist, gilt es den konkreten Prozess, inklusive einer eventuellen Übergangszeit, zu planen und umzusetzen. Die Ergänzung von \ac{aml} wurde bereits beim Entwurf der BI-Architektur vorgesehen. Daher ist es naheliegend, dass diese zusätzlich umgesetzt wird, um anschließend das Reporting um fortgeschrittene Analysen zu erweitern. Darüber hinaus bietet es sich an, Schulungen in \textit{Power BI} zu organisieren, damit das volle Potenzial vom Self-Service-Reporting genutzt werden kann.

Die Azure Dienste werden stetig weiterentwickelt und verbessert \cite{chilberto_building_2020}, was sich zukünftig positiv auf die vorgestellte Architektur auswirken wird. Insbesondere ist hervorzuheben, dass \textit{Power BI} und \textit{Purview} in der Vorschauphase bereits die Möglichkeit bieten, direkt auf eine Datenquelle in einem virtuellen Netzwerk zuzugreifen \cite{msdoc_22_purview_manPE, msdoc_22_pbi_vnetGateway}. Wenn diese Funktionen veröffentlicht werden, können damit die \acp{vm} ersetzt werden, wodurch Kosten und Komplexität des BI-Systems verringert werden.